\chapter{Introducción}
\section{Introducción: Epidemiología y salud}
La Epidemiología es la disciplina científica que estudia la frecuencia y distribución de la enfermedad en la población y sus determinantes.
\begin{table}[H]
	\centering
	\begin{tabular}{M{2.5cm}N{13.5cm}}
		\rowcolor{black}\textcolor{white}{\textbf{Término}}&\textcolor{white}{\textbf{Definición}}\\
		Estudio&Incluye: vigilancia, observación. test de hipótesis, investigación analítica y experimentación.\\
		\rowcolor{hiperlightgray}Distribución&Referencia a análisis de: tiempos, personas, sitios y clases sociales.\\
		Determinantes&Factores que influyen en la salud: biológicos, químicos, físicos, sociales, culturales, económicos, genéticos y comportamentales\\
		\rowcolor{hiperlightgray}Estados y eventos relacionados con la salud&Referencia a: enfermedades, causa de la muerte, comportamiento (tabaquismo, $\dots$), estado de salud positivos, usod de servicios sanitarios,$\dots$\\
		Poblaciones específicas& Incluye a aquellas con características diferenciales como la ocupación de los grupos.\\
		\rowcolor{hiperlightgray}Aplicación de prevención y control&Objetivos de salud pública para promover, proteger o restaurar la salud.\\
		\hline
	\end{tabular}
	\caption{Terminología útil en epidemiología.}
\end{table}

Frente a la medicina clínica, que estudia individualmente al paciente, la epidemiología estudia a la población en su conjunto. La epidemiología parte de la premisa de que la enfermedad y los problemas de salud no se distribuyen aleatoriamente en una población, sino que existen, en cada individuo, unos características que predisponen a la enfermedad o protegen de ella, pudiendo ser factores genéticos y/o exposición a ciertos riesgos sociales y ambientales. La epidemiología permite:
\begin{itemize}[itemsep=0pt,parsep=0pt,topsep=0pt,partopsep=0pt]
	\item Estudio histórico de la enfermedad: monitorización o vigilancia de tendencias temporales de la enfermedad.
	\item Identificación de síndromes.
	\item Diagnóstico de salud de la comunidad.
	\item Investigación etiológica: identificación de factores que llevan de la salud a la enfermedad.
	\item Evaluación de servicios y de intervenciones sanitarias como el tratamiento médico y campañas de salud pública. Llevan de la enfermedad a la salud.
\end{itemize}

Dentro de la epidemiología se pueden distinguir:
\begin{itemize}[itemsep=0pt,parsep=0pt,topsep=0pt,partopsep=0pt]
	\item \textbf{Epidemiología descriptiva}: estudia la frecuencia y distribución de las enfermedades y problemas de salud, observando los hechos básicos de la distribución en términos de persona, lugar y tiempo. Se basa en estudios descriptivos, individuales (descripción de un caso o series de casos) o poblaciones de región geográfica o administrativa (estudios ecológicos y encuestas).
	\item \textbf{Epidemiología analítica}: estudia los determinantes de la enfermedad, pone a prueba una hipótesis acerca de una relación entre causa y enfermedad, relacionando la exposición de interés con la enfermedad de interés. Se articula en torno a estudios analíticos de tipo observacional (estudios transversales (encuestas), estudios de cohortes y de casos-controles) y experimentales (ensayos clínicos o ensayos de intervención).
\end{itemize}

El término salud se puede definir como: Estado de completo bienestar físico, mental y social y no sólo como ausencia de enfermedad (OMS, 1948); recurso de la vida diaria, concepto que aúna recursos sociales y personales además de aptitudes físicas (Carta de Ottawa, 1986); o como ausencia de enfermedad.
\section{Diagnóstico, pronóstico y tratamiento}
La epidemiología resulta también fundamental no sólo para la salud pública sino también en la práctica clínica. La práctica de la medicina depende de datos poblacionales. El médico aplica un modelo de probabilidad basado en poblaciones al paciente a tratar. Los conceptos y datos basados en la población subyacen a los procesos fundamentales de la práctica clínica: <<diagnóstico + pronostico + tratamiento>>.

Los datos disponibles de enfermedad pueden ser útiles para indicar un diagnóstico, aunque no sea concluyente. El médico aplica un modelo de probabilidad al paciente basado en poblaciones. Los pronósticos de tiempo de vida o enfermedad se basan en la experiencia del médico con un número muy amplio de pacientes que han tenido la misma enfermedad, a los que se observó en el mismo estadio de la enfermedad y que recibieron el mismo tratamiento, es decir, se basa en datos poblacionales.

La selección del tratamiento adecuado se basa en la población. Los ensayos clínicos con distribución aleatoria que investigan los efectos de un tratamiento en un grupo grande de pacientes son el medio ideal para identificar el tratamiento más eficaz y clinicamente  adecuado. Aunque parezca que prevención y tratamiento son actividades excluyentes, la prevención es una parte integral de salud pública y práctica clínica. La epidemiología sirva para planificar y poner en marcha programas de prevención, o para dirigir investigaciones clínicas que ayudan a controlar la enfermedad.

Ejemplos de usos de la epidemiología en la práctica clínica son el descubrimiento de la vacuna frente a la viruela por Edward Jenner, al descubrir que las vaqueras en contacto con vacas infectadas de la viruela bovina no contraian la enfermedad; o el descubrimiento del modo de transmisión de \textit{Vibrio Cholerae} por John Snow, patógeno de la cólera, a través de aguas infectadas, al apreciar que disminuyen los casos en Londres por el cambio de lugar de extracción de aguas (es un caso de expermiento natural, el investigador no modifica la variable, lo hace el ambiente).
\section{Sistema de información sanitaria}
Los sistemas de información sanitaria son un conjunto ordenado de datos organizados con objeto de suministrar información adecuada para informar del estado de salud, de las necesidades sanitarias y dar apoyo a las intervenciones en salud pública. Sirven para valorar la situación sanitaria poblacional, tener sistemas de vigilancia de salud, establecimiento de programas de salud e investigación.

\begin{table}[H]
	\centering
	\begin{tabular}{M{2.5cm}N{4cm}N{5.25cm}N{4cm}}
		\rowcolor{black}\textcolor{white}{\textbf{}}&\textcolor{white}{\textbf{Registros}}&\textcolor{white}{\textbf{Encuestas}}&\textcolor{white}{\textbf{Sist. Notificaciones}}\\
		Base poblacional&
			\begin{tabular}{N{4cm}}
				Natalidad\\
				Mortalidad\\
				Muertes fetales\\
				Enfermedades\\
			\end{tabular}&
			\begin{tabular}{N{5.25cm}}
				Encuesta de salud\\
				Encuesta Nacional de Salud\\
				Encuesta de Población Activa\\
			\end{tabular}&
			\begin{tabular}{c}
				CMBD\\
			\end{tabular}\\
		\rowcolor{hiperlightgray}Procedentes de servicios sanitarios&CMBD&&EDO\\
		\hline
	\end{tabular}
	\caption[Registros de información sanitaria]{Registros de información sanitaria: tipos de registros y la información que dan. Siglas: \textit{CMBD}: Conjunto Mínimo Básico de Datos; \textit{EDO}: Enfermedad de Declaración Obligatoria.}
\end{table}

Los registros de enfermedades son entidades encargadas de recoger, almacenar, analizar e interpretar los datos sobre personas con una determinada enfermedad. El objetivo de estos registros en una población es conocer el número de casos diagnosticados en un periodo definido de tiempo y residentes en el área geográfica abarcada. Los datos de utilidad para el control de enfermedades se usan en etiología, prevención primaria y secundaria, planificación sanitaria y atención al paciente.

La información de los registros procede de todos los centros sanitarios existentes en la población. La detección de los casos se realiza a través de la información recibida con periodicidad en el registro procedente de servicios de documentación o laboratorios de anatomía patológica. Es de gran utilidad esta información, así como la facilitada por otros servicios hospitalarios en los que se diagnostica o trata estas enfermedades.