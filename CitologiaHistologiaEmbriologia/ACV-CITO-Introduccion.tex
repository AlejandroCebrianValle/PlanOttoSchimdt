\chapter{Introducción y teoría celular}
\section{Teoría celular}
La teoría celular fue elaborada por tres grandes científicos como J.M Schleiden (enunciándola en 1938), T. Schwann (1939) y R. Virchow (1955) que establece tres elementos fundamentales:
\begin{itemize}[itemsep=0pt,parsep=0pt,topsep=0pt,partopsep=0pt]
    \item La célula es la unidad funcional y estructural de los seres vivos.
    \item Todos los seres vivos está compuestos por una o más células.
    \item Toda célula proviene de otra célula\footnote{\textit{omnis cellula ex cellula}, enunciado de Virchow.}.
\end{itemize}
\section{Organización celular}
Desde el punto de vista de la organización, las células se clasifican en dos grupos, a saber, procariotas y eucariotas, con notables diferencias:
\begin{table}[H]
    \begin{tabular}{N{3.25cm}M{6cm} M{6.9cm}}
        \rowcolor{black}\textcolor{white}{\textbf{}}&\textcolor{white}{\textbf{Procariotas}}&\textcolor{white}{\textbf{Eucariotas}}\\
        Organización celular&Unicelular&Unicelular o pluricelular\\
        \rowcolor{hiperlightgray}Tamaño (aprox)&1 a 10 $\mu$m&10 a 50 $\mu$m\\
        Envoltura nuclear&Ausente&Doble\\
        \rowcolor{hiperlightgray}Compartimentación&No&Sí\\
        Orgánulos&No&Sí (Retículo endoplásmico, Aparato de Golgi, Mitocondrias, Lisosomas, Peroxisomas)\\
        \rowcolor{hiperlightgray}Citoesqueleto&No&Sí\\
        Genoma&Sin histonas, circular, citoplasmático&Con histonas y otras proteínas, Lineal, nuclear\\
        \rowcolor{hiperlightgray}Nucléolo&No&Sí\\
        Ribosomas&70S (50S y 30S)&80S (60S y 40 S)\\
        \hline
    \end{tabular}
    \caption[Diferencias entre procariotas y eucariotas]{Diferencias entre procariotas y eucariotas.}
\end{table}
\subsection{Células eucariotas}
Las células eucariotas se dividen en compartimentos principales:
\begin{itemize}[itemsep=0pt,parsep=0pt,topsep=0pt,partopsep=0pt]
    \item \textbf{Citoplasma}: parte localizada fuera del núcleo. Contiene orgánulos e inclusiones en un gel acuoso concentrado denominado citosol o matriz citoplasmática. La matriz está compuesta por un 70 a 80 \% de agua y solutos desde iones inorgánicos, metabolitos intermedios, glúcidos, lípidos, proteínas y ARN.
    \item \textbf{Núcleo}: el núcleo es un compartimento limitado por una membrana (envoltura nuclear) que contiene al material genético (ADN) junto con la maquinaria necesaria para la replicación y su transcripción y procesamiento a ARN.
    \item\textbf{Protoplama}: constituye todo el interior celular, sumando citoplasma y núcleo.
\end{itemize}
\chapter{Membrana plasmática}
La membrana plasmática es una barrera semipermeable que separa el medio extracelular del intracelular, así como la delimitación entre orgánulos celulares y el citoplasma (llamadas entonces membranas citoplasmáticas). Debido a que actúa como barrera selectiva al paso de las moléculas, la membrana plasmática determina la composición del citoplasma. Dado su pequeño tamaño (8 a 10 nm), no es visible con el microscopio óptico.

En la actualidad, el modelo más aceptado es el propuesto por Singer y Nicolson (1972), llamado \textbf{modelo de mosaico fluido}:
\begin{itemize}[itemsep=0pt,parsep=0pt,topsep=0pt,partopsep=0pt]
    \item La membrana está formada por lípidos, proteínas y glúcidos que se disponen en una configuración estable de baja energía.
    \item Los lípidos se orientan formando una doble capa lipídica en la que se disponen proteínas que interaccionan entre sí y con los lípidos, con capacidad limitada de movimiento lateral.
    \item Las membranas biológicas son estructuras asimétricas en cuanto a la distribución de sus componentes, ya que los glúcidos se localizan exclusivamente en la cara externa de la membrana plasmática.
\end{itemize}
\section{Composición química}
En la mayoría de los tipos celulares, la membrana plasmática se compone de un 50 \% de lípidos y otro tanto de proteínas en peso. Como las proteínas son mucho más grandes, este porcentaje corresponde más o menos a una molécula de proteína por cada 50 a 100 de lípidos. Algunas membranas varían estos porcentajes, como la mitocondrial (75 \% de proteínas) o la vaina de mielina (sólo 20\% de proteínas).
\subsection{Lípidos de membrana}
Los lípidos tienen carácter anfipático y cuando se encuentran en un medio acuoso se orientan formando una bicapa, enfrentando los grupos hidrófilos al medio acuoso y los hidrófobos en el centro. Entre estos, se encuentran tres grupos de familias químicas:
\subsubsection{Fosfolípidos}
Contienen un residuo de ácido fosfórico mediante enlace éster. Se dividen en:
\begin{itemize}[itemsep=0pt,parsep=0pt,topsep=0pt,partopsep=0pt]
    \item \textbf{Fosfoglicéridos}: son los lípidos más abundantes, compuestos por glicerol esterificado en sus posiciones 1 y 2 (diacilglicerol) con ácidos grasos y un grupo fosfato (posición 3). Los principales son la fosfatidilcolina (lecitina), fosfatidilserina, fosfatidiletanolamina (cefalina), fosfatidilinositol, fosfatidilglicerol y difosfatidilglicerol (cardiolipina). La fosfatidilcolina es el fosfoglicérido principal de las membranas plasmática y citoplasmáticas; y la cardiolipina el de las mitocondriales.
    \item\textbf{Esfingolípidos}: compuestos derivados de la esfingosina (o ceramida, 2-amino-4-octadecen-1,3-diol) . La esfingosina unida a fosfocolina o fosfoetanolamina forman las esfingomielinas, abundantes en la vaina de mielina.
\end{itemize}
\subsubsection{Glucolípidos}
La ceramida unida a glúcidos forma los glucoesfingolípidos, que se localizan en la cara externa de la membrana plasmática. Pueden ser:
\begin{itemize}[itemsep=0pt,parsep=0pt,topsep=0pt,partopsep=0pt]
    \item \textbf{Cerebrósidos}: la ceramida se una a un monosacárido, ya sea galactosa (galactocerebrósidos), en las membranas del tejido nervioso; o a glucosa (glucocerebrósidos), en tejidos no nerviosos.
    \item\textbf{Gangliósidos}: la ceramida se une con oligosacáridos con al menos un residuo de ácido siálico (N-acetilneuramínico), con carga negativa. Abundantes en células nerviosas.
\end{itemize}
\subsubsection{Colesterol}
El colesterol es un lípidos de unos 27 carbonos que representa el 40 \% de los lípidos de membrana, encargado de regular la fluidez de ésta.
\subsubsection{Movimientos de los lípidos}
Los componentes de la membrana tienen una posibilidad de movimiento limitada que le permite cierta fluidez a ésta. Pueden ser:
\begin{itemize}[itemsep=0pt,parsep=0pt,topsep=0pt,partopsep=0pt]
    \item \textbf{Difusión lateral}: las moléculas se desplazan en 2 dimensiones siguiendo el plano de la membrana.
    \item\textbf{Rotación}: la molécula puede girar sobre su eje mayor.
    \item\textbf{Flexión}: las cadenas de ácidos grasos pueden flexionarse o doblarse dentro de la doble capa.
    \item\textbf{Flip-Flop}: movimiento en el que una molécula se transloca (normalmente por acción de una enzima translocasa) de una monocapa a otra.
\end{itemize}
\subsubsection{Lípidos y fluidez}
La fluidez de las membranas depende de los lípidos y es una característica fundamental para la transducción de señales. Depende a su vez de factores como la temperatura, concentración de colesterol y naturaleza de los fosfolípidos y de la longitud y grado de insaturaciones de los ácidos grasos.

La fluidez de incrementa con la temperatura, la presencia de lípidos insaturados y de cadena corta y la falta de colesterol. El colesterol se inserta en la bicapa con sus grupos apolares próximos a la cabeza de los fosfolípidos, interfiriendo con el movimiento de las cadenas de ácidos grasos, así como impedir un gran empaquetamiento en las cadenas de estos en caso de bajas temperaturas.
\subsubsection{Distribución asimétrica}
Existe una mayor proporción de fosfatidilcolina y esfingomielina y esterificaciones de ácidos grasos saturados en la hemimembrana E o exoplasmática, y mayor cantidad de fosfatidilcolina y fosfatidilserina (es marcador de apoptosis y señal para macrófagos para que la fagociten) y esterificaciones ácidos grasos insaturados, en la hemimembrana P o citoplasmática. Debido a su mayor contenido en ácidos grasos insaturados, la hemimembrana P es más fluida.

En la membrana pueden encontrarse regiones de lípidos implicadas en la señalización celular, denominándose microdominios o balsas lipídicas (\textit{rafts}), teniendo mayor riqueza en esfingolípidos y colesterol. Los esfingolípidos poseen largas cadenas de ácidos grasos saturados, lo que hace que las balsas lipídicas sean más espesas y menos fluidas que el resto de la membrana. Las caveolas están estrechamente relacionadas con estos \textit{rafts} y son invaginaciones de membrana recubiertas de la proteína caveolina, mediadora en la endocitosis.
\subsection{Proteínas de membrana}
Las proteínas son características de cada tipo celular y confieren a la membrana sus funciones biológicas. Su distribución es asimétrica y según su disposición en la membrana se habla de proteínas integrales (intrínsecas) o periféricas (extrínsecas).
\subsubsection{Proteínas integrales}
Las proteínas integrales están firmemente unidas a lípidos por interacciones hidrofóbicas, por lo que sólo se disocian de éstos mediante tratamientos drásticos que destruyen la integridad de la membrana como detergentes. Una de sus funciones principales es la recepción de ligandos.

Las proteínas integrales suelen atravesar la membrana por completo, siendo proteínas transmembrana. Si sobresalen por ambos lados de la doble membrana, se denominan monopaso, bitópicas o de paso único. Si emerge varias veces, se habla de proteínas multipaso, politópicas o de paso múltiple. Las proteínas que se anclan a la membrana pero no la atraviesan se denominan monotópicas. Su orientación con respecto a las hemimembranas  se determina en el retículo endoplásmico.
\subsubsection{Proteínas periféricas}
Las proteínas periféricas sobresalen solo por uno de los lados de la membrana celular y no se asocian a lípidos, por lo que pueden extraerse sin necesidad de alterar la bicapa. Este tipo de proteínas se pueden unir a una proteína integral, a un lípido o a un oligosacárido unido a un lípido.

Al igual que los lípidos, tienen movimiento de rotación y difusión, pero no de translocación.
\subsection{Glúcidos de membrana}
Están representados en su mayoría por oligosacáridos unidos covalentemente a segmentos o dominios de proteínas o lípidos, formando, en cada caso, glucoproteínas o glucolípidos. Su distribución es exclusiva de la hemimembrana E, constituyendo la cubierta celular o glucocálix, en la que pueden encontrarse algunas proteínas.

Los oligosacáridos unidos a proteínas pueden unirse a asparagina (N-oligosacáridos) o a treonina o serina (O-oligosacáridos). Los que están unidos a lípidos se unen fundamentalmente a esfingolípidos y en menor medida al fosfatidilinositol.

Las principales funciones son:
\begin{itemize}[itemsep=0pt,parsep=0pt,topsep=0pt,partopsep=0pt]
    \item Presentan propiedades inmunológicas, ya que los glúcidos conforman el sistema ABO y MN.
    \item Interviene en los fenómenos de reconocimiento celular, importantes durante el desarrollo embrionario.
    \item Contribuye al reconocimiento y fijación de sustancias a incorporar mediante pinocitosis o fagocitosis.
    \item Participa en las uniones celulares entre sí y con la matriz extracelular.
    \item Es responsable (por el ácido siálico) de la carga negativa de la superficie celular.
\end{itemize}
\subsection{Renovación de la membrana}
La membrana plasmática se encuentra en un continuo proceso de reciclaje debido a procesos de endocitosis y exocitosis. La renovación se realiza a partir de membranas del aparato de Golgi, procedentes a su vez del retículo endoplásmico, lugar donde se sintetizan las membranas citosólicas y plasmáticas pero no de mitocondria ni cloroplasto. 
\subsection{Membrana eritrocitaria}
Las proteínas de membrana más estudiadas con las del eritrocito dada la facilidad para obtenerlas. Se compone, además de la cantidad antes descrita de lípidos y glúcidos, de:
\begin{itemize}[itemsep=0pt,parsep=0pt,topsep=0pt,partopsep=0pt]
    \item \textbf{Glucoforina}: glucoproteína transmembrana cuyo segmento externo se une a oligosacáridos, principalmente ácido siálico.
    \item\textbf{Proteína de la banda 3}: proteína transmembrana de paso múltiple que posee alrededor de 14 segmentos transmembrana que forman un canal para el transporte antiparalelo de \ch{Cl^-} y \ch{HCO^{3-}}.
    \item\textbf{Espectrina}: principal proteína periférica de la membrana del eritrocito, es un tetrámero constituido por dos cadanas $\alpha$ y dos cadena $\beta$ que se asocian con filamentos de actina para formar una red bajo la membrana plasmática.
    \item\textbf{Anquirina}: forma un nexo de unión entre el dominio citoplásmico de la proteína de la banda 3, la actina y la espectrina.
    \item\textbf{Proteína 4.1}: fija y une el dominio citoplásmico de la glucoforina junto con la red de actina y espectrina.
\end{itemize}







