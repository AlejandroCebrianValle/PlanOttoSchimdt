\chapter{Transporte a través de membranas}
\section{Transporte de moléculas pequeñas}
El transporte de moléculas pequeñas a través de membrana no modifica su morfología y se lleva a cabo sin la participación del citoesqueleto. Puede ocurrir mediante los siguientes mecanismos.
\begin{table}[H]
    \begin{tabular}{M{3.5cm}M{4cm}N{3.5cm} N{4cm}}
        \multirow{6}{3.5cm}{Dependencia de energía} & \multirow{3}{4cm}{\textbf{Transporte pasivo}\\
             (a favor de gradiente, sin consumo de energía)} & \multicolumn{2}{N{7cm}}{Difusión pasiva o  simple} \\
        &  & \multirow{2}{3.5cm}{Difusión facilitada} & Proteínas transportadoras \\
        &  &  & Proteínas canal \\
        & \multirow{3}{4cm}{\textbf{Transporte activo}\\(en contra de gradiente, consumo de energía)} & \multirow{2}{4cm}{Primario (hidrolisis de ATP)} & ATPasa de \ch{Na^+}/\ch{K^+}; ATPasa de \ch{Ca^{2+}} \\
        &  &  & Transportadores ABC \\
        &  & \multicolumn{2}{N{7cm}}{Secundario (energía en gradiente iónico)} \\
        \hline
        \multirow{2}{3.5cm}{Dirección de los compuestos transportados} & \multirow{2}{4cm}{Uniporte o transporte simple} & \multicolumn{2}{N{7cm}}{Cotransporte} \\
        &  & Paralelo o simporte & Antiparalelo o antiporte
    \end{tabular}
    \caption[Modalidades de transporte a través de membrana\label{tab:CITO:Transporte:ModalidadTransporte}]{Modalidades de transporte a través de membrana.}
\end{table}
\subsection{Transporte pasivo}
El transporte pasivo de moléculas se efectúa a favor de gradiente de concentración o carga eléctrica, por lo que no se produce gasto de energía. Se puede diferenciar entre una forma simple o facilitada.
\subsubsection{Difusión pasiva o simple}
El mecanismo más sencillo mediante el que las moléculas pueden atravesar la membrana plasmática es la difusión pasiva: una molécula se disuelve en la bicapa lipídica, difunde a través de ella y después se disuelve en la solución acuosa al otro lado de la membrana. Gracias a este mecanismo, las sustancias atraviesan esta membrana desde la zona más concentrada a la más diluida hasta igualar concentraciones a ambos lados.

La membrana plasmática deja pasar con facilidad pequeñas moléculas no polares (\ch{O_2}, \ch{N_2}, benceno), y moléculas pequeñas polares sin carga (agua, urea, glicerol, \ch{CO_2}). El agua se mueve con más facilidad y se desplaza hacia donde los solutos están más concentrados mediante un proceso de ósmosis.
\subsubsection{Difusión facilitada}
Las moléculas se desplazan a favor de gradiente de concentración y, en el caso de moléculas cargadas, de potencial eléctrico. La difusión facilitada se diferencia en que las moléculas no se disuelven en la bicapa lipídica y necesitan la participación de una proteína transportadora para atravesarla. Existen dos clases de proteínas, las transportadoras y las que forman canales.
\begin{itemize}[itemsep=0pt,parsep=0pt,topsep=0pt,partopsep=0pt]
    \item \textbf{Proteínas transportadoras}: Las proteínas transportadoras (carriers o permeasas) son proteínas integrales de membrana que sufren un cambio conformacional durante el transporte. Muy selectivas, a menudo sólo transportan un tipo de molécula. Permiten el paso de determinadas moléculas o iones, saturándose el transporte a determniadas concentraciones. La cinética del transporte es similar a la de las reacciones enzimáticas. Son responsables de la difusión de glúcidos, nucleótidos y aminoácidos.
    \item\textbf{Proteínas canal}:
\end{itemize}
