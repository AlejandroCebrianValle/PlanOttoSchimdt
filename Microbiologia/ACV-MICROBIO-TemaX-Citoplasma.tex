\chapter{Ultraestructura interna}
\section{Citoplasma}
El citoplasma está compuesto en un 70\% por agua, dónde se encuentran en solución coloidal una gran cantidad de macromoléculas u otras biomoléculas o bioelementos, con una presión oncótica u osmótica de 2 atmósferas. En el citoplasma, además, se encuentran presentes las enzimas bacterianas necesarias para su metabolismo y chaperonas (proteínas que ayudan a una proteína a alcanzar su plegamiento correcto). Así mismo se hallan proteínas con una función estructural, que dan forma a la célula, asimilándose al citoesqueleto de eucariotas. Algunas proteínas son:
\begin{itemize}[itemsep=0pt,parsep=0pt,topsep=0pt,partopsep=0pt]
	\item \textbf{FtsZ} (similar a la tubulina): tienen función en la división celular. Se encuentra en numerosas especies de Bacteria y Archaea.
	\item \textbf{MreB} (similar a la actina): dan la forma a la célula. Se encuentran en muchos géneros de bacilos.
	\item \textbf{Crescentina}: (similar a filamentos internos) dan forma a la célula. Se encuentra en especies como el \textit{Caulobacter crecentus}.
\end{itemize}

En el citoplasma se encuentran, además, orgánulos y el nucleoide. El nucleoide es el área del citoplasma donde se coloca el material genético (normalmente, un cromosoma circular) y plásmidos. Estos últimos son cadenas de DNA no vitales, pero que confieren al individuo ventajas adaptativas como resistencia a antibióticos, a daños or metales pesados o la incorporación al metabolismo de otros compuestos, como el plásmido \textit{Tol} (tolueno) de \textit{Pseudomona}. El cromosoma se encuentra, como en eucariontes, superenrollado, pero en este proceso no participan histonas.

Los ribosomas son el único orgánulo celular compartido con los tres dominios celulares. Con la misma función en todos los individuos, el bacteriano es de 70S, con una subunidad menor de 30S (compuesto por 21 proteinas y RNAr de 16S) y una mayor de 50S (compuesto por 34 proteínas y dos RNAr, uno de 23S y otro de 5S).

Otro tipo de orgánulo bacteriano son las inclusiones. Las inclusiones son gránulos de material orgánico o inorgánico envuelto o no por una membrana no unitaria (formadas por proteínas, lipoproteínas o glicoproteínas). Se generan, normalmente, bajo condiciones determinadas de crecimiento. Pueden ser:
\begin{itemize}[itemsep=0pt,parsep=0pt,topsep=0pt,partopsep=0pt]
	\item \textbf{Orgánicas}: de polisacáridos, de poli-$\beta$-hidroxialcano o cianoficina
	\item \textbf{Inorgánicas}: polifosfato o de azufre elemental. 
\end{itemize}

Algunos otros orgánulos no rodeados por membrana son las vesículas de gas, clorosomas, carboxisomas o magnetosomas. Como única excepción a la no presencia de orgánulos rodeados por membrana unitaria son los tilacoides de las cianobacterias.
\subsection{Inclusiones orgánicas}
Pueden ser:
\begin{itemize}[itemsep=0pt,parsep=0pt,topsep=0pt,partopsep=0pt]
	\item \textbf{Polisacarídicas}: de glucógeno (ósido de reserva animal) o de almidón (ósido de reserva vegetal). Ambos polímeros de glucosasirven como reservorio de carbono. Se comienzan a sintetizar cuando en el medio hay falta de nitrógeno. Cuando se restituye este elemento, se forman las proteínas con el carbono almacenado.
	\item De \textbf{poli-$\beta$-hidroxialcanoatos} (PHA): se encuentran entre ellos:
	\begin{itemize}[itemsep=0pt,parsep=0pt,topsep=0pt,partopsep=0pt]
		\item \textbf{Poli-$\beta$-hidroxibutirato} (PHB) polímero de ácido 3-hidroxibutírico unidos por enlace éster. Sirve como reservorio de carbono antes de la esporulación en \textit{Pseudomona} y \textit{Bacillus}.
		\item \textbf{Otros}, con función de bioplásticos (en el organismo vivo se utilizan como reserva de carbono), son copolímeros de poli-$\beta$-hidroxibutírico y poli-$\beta$-hidroxivalérico.
	\end{itemize}
	\item \textbf{Cianoficina}: presente únicamente en cianobacterias, como la Anaystis nidulans, son polímeros de aspartato con ramificaciones de arginina, unidos por enlaces peptídicos. Estas inclusiones, cuya generación se produce ante la falta en el medio de algún nutriente, se utilizan como reserva de carbono y nitrógeno.
\end{itemize}
\subsection{Inclusiones inorgánicas}
De las más conocidas son:
\begin{itemize}[itemsep=0pt,parsep=0pt,topsep=0pt,partopsep=0pt]
	\item \textbf{Inclusiones de polifosfato}: se dan en los géneros Lactobacillus y Spirilum. Se conocen también como gránulos de volutina o metacromáticos. Sirven de reserva de fosfato, produciéndose ante la falta de azufre.
	\item \textbf{Inclusiones de azufre}: observables en las bacterias verdes y en las quimiolitotrofas. En ambos casos, son acumulaciones temporales de azufre elemental que se da por un proceso lento de oxidación a sulfato ($SO_4^{2-}$). Se usa, en el caso de bacterias verdes, para obtener poder reductor, y en el caso de quimiolitotrofos, para obtener poder reductor y energía.
\end{itemize}
\subsection{Vacuolas de gas}
Estos orgánulos no presentan límite de membrana unitaria. Se hallan presentes en algunas procariotas, sobre todo acuáticas, siendo su función la de permitir al organismo moverse en la vertical hacia hábitats más propicios (más luz, nutrientes,$\dots$). Son agregados de un gran número de estructuras pequeñas huecas cilíndricas o fusiformes. Las cianobacterias con vesículas de gaspresentan una membrana especial, permeable a los gases, pero no al agua. De esta forma, al reducir su contenido en gas, desciende en profundidad; y al aumentar la cantidad de gas, ascienden.