\chapter{\textit{Phylum Plathelmintes}}

Los trematodos son una clase de gusanos planos (Platyhelmintes) de simetría bilateral, cuerpo aplanado dorsiventralmente, foliáceo, con aparato digestivo bifurcado, sin aparato respiratorio ni circulatorio, mayoritariamente hermafroditas y con ventosas como órgano de fijación.

Diastema hace referencia a la posesión de dos ventosas (oral y ventral) de estos géneros.
\newpage
\section{Clase \textit{Cestoda}}
\subsection{\textit{Taeniarrynchus}}
Los tenidos son una familia incluida dentro da la clase Cestoda, una división del phylum Platyhelminthes. Estas especies cuentan con una serie de características comunes:
a) Son gusanos planos de forma alargada o de cinta con segmentación corporal.
b) Ausencia de aparato digestivo. Se alimentan a través del tegumento.
c) La mayoría son hermafroditas.
d) El cuerpo está dividido en tres partes perfectamente distinguibles:
• Escólex: región más anterior, en él se encuentran los órganos de fijación.
• Cuello: región proliferativa de los distintos anillo o proglótides.
• Estróbilo: conjunto de los proglótides.

Morfología
Taeniarrynchus saginatus

Taeniarrynchus saginatus, o Taenia saginata, es un gusano parásito del intestino humano en su forma adulta y de la musculatura de otros animales en su fase larvaria. De hasta 12 metros de longitud, llega tener hasta 2000 anillos formando su estróbilo. 

Presenta un escólex inerme (róstelo sin ganchos) con 4 ventosas grandes y poco salientes. Este escólex tiene forma de priesma rectangular, siendo más ancho que alto. Sigue a esta estructura el cuello, donde no se pueden distinguir estructuras con las tinciones de rutina, siendo la zona de proliferación de los proglótides. Los anillos del estróbilo se dividen en dos tipos:
• Maduros: forman la parte más proximal del estróbilo. Planos, cuadrangulares, poseen de 300 a 400 testículos. Los anillos poseen un único poro genital, que se abre de forma alterna e irregular a un lado u otro. A este salen, por un lado, la vagina con esfínter), que asciende hasta bifurcarse en el ootipo (donde acontece la fecundación), en la glándula vitelógena (estructura más inferior del proglótide) y en el ovario, que es bilobulado (superior a las estructuras anteriores, ambos lóbulos son iguales). Sobre el ootipo surge el útero, ramificado a ambos lados, con de 15 a 30 pares de ramas uterinas. Del poro genital también surge el vaso deferente, que termina en la bolsa del cirro, de donde surge el cirro o pene, y que comienza en las vesículas testiculares por fusión de sendos vasos deferentes. En su exterior se encuentran, de exterior a interior, y en ambos lados, un nervio y el canal excretor, que se une por medio de una unión transversal (basal al proglótide) con el del otro lado.
• Grávido: tan solo se halla el útero, repleto de huevos. El resto de estructuras han degenerado.

Otras fases larvarias son:
• Huevo: esférico, sin opérculo ni espolón, presentan una doble corona radiada (embióforo) con un embrión (oncosfera) completamente desarrollado, con tres pares de ganchos (hexacanto). Fase de eliminación en el hospedador definitivo.
• Cisticerco o metacestodo: presente en la musculatura del hospedador intermediario, se trata de una vesícula de ligero color rosado (mioglobina) con el escólex invaginado.

Taenia solium

Taenia solium, o “solitaria”, es un gusano parásito del intestino humano en su forma adulta y de la musculatura de otros animales en su fase larvaria. De hasta 7 metros de longitud, llega tener hasta 700 o incluso 1000 anillos formando su estróbilo. 

Presenta un escólex armado (róstelo con doble corona de ganchos en forma de “navaja” o “pinza de cangrejo”) con 4 ventosas grandes y poco salientes. Este escólex tiene forma piriforme. Sigue a esta estructura el cuello, donde no se pueden distinguir estructuras con las tinciones de rutina, siendo la zona de proliferación de los proglótides. Los anillos del estróbilo se dividen en dos tipos:
• Maduros: forman la parte más proximal del estróbilo. Planos, cuadrangulares, poseen de 100 a 150 testículos. Los anillos poseen un único poro genital, que se abre de forma alterna e irregular a un lado u otro. A este salen, por un lado, la vagina (sin esfínter), que asciende hasta bifurcarse en el ootipo (donde acontece la fecundación), en la glándula vitelógena (estructura más inferior del proglótide) y en el ovario, que es trilobulado (superior a las estructuras anteriores, con dos lóbulos iguales y uno tercero muy poco perceptible). Sobre el ootipo surge el útero, ramificado a ambos lados, con de 7 a 14 pares de ramas uterinas. Del poro genital también surge el vaso deferente, que termina en la bolsa del cirro, de donde surge el cirro o pene, y que comienza en las vesículas testiculares por fusión de sendos vasos deferentes. En su exterior se encuentran, de exterior a interior, y en ambos lados, un nervio y el canal excretor, que se une por medio de una unión transversal (basal al proglótide) con el del otro lado.
• Grávido: tan solo se halla el útero, repleto de huevos. El resto de estructuras han degenerado.

Otras fases larvarias son:
• Huevo: esférico, sin opérculo ni espolón, presentan una doble corona radiada (embrióforo) con un embrión (oncosfera) completamente desarrollado, con tres pares de ganchos (hexacanto). Fase de eliminación en el hospedador definitivo.
• Cisticerco o metacestodo: presente en la musculatura del hospedador intermediario, se trata de una vesícula de ligero color rosado (mioglobina) con el escólex invaginado.


Ciclo biológico
Taenia saginata

Parásito heteroxeno, tiene un hospedador definitivo (el ser humano) y uno intermediario (ganado bovino). Es cosmopolita.

El ciclo comienza con la fecundación de los proglótides (mediante autofecundación o entre proglótides distintos), pasando el anillo maduro a su estado grávido, con el consiguiente aumento del útero y la degeneración del resto de estructuras. Tras ello, se desprende del estróbilo (tras 2 o tres meses se observan formas de eliminación) de uno en uno, pudiendo forzar el esfínter anal en su salida. 

En el exterior, por la temperatura y la labilidad del proglótide, se rompe y libera los huevos. Si en esa zona está el hospedador intermediario (por ejemplo, pastos de los que se alimenta el ganado bovino) puede ingerir los huevos, contaminándose. La cubierta radiada les protege del ácido estomacal, rompiéndose el embrióforo en el duodeno, surgiendo la oncosfera. La oncosfera, por medio de sus ganchos rompe la pared intestinal y busca un capilar, por el que viajará normalmente a la musculatura del hospedador. Allí acontece el proceso de vesiculización: el parásito entra en una de esas células, transformándose en un cisticerco (Cysticercus bovis). Cuando el hospedador definitivo ingiere la carne contaminada, este pasa por el estómago sin problemas y llega al duodeno, donde se evagina el escólex del cisticerco, rompiéndose la vesícula que lo protegía, y se fija a la mucosa intestinal. Allí crece, en un proceso conocido como estrobilización, donde llega a su forma definitiva de adulto, pudiéndose cerrar el ciclo.

Taenia solium

El ciclo comienza con la fecundación de los proglótides (mediante autofecundación o entre proglótides distintos), pasando el anillo maduro a su estado grávido, con el consiguiente aumento del útero y la degeneración del resto de estructuras. Tras ello, se desprende del estróbilo (tras 2 o tres meses se observan formas de eliminación) en segmentos de 3 a 6 anillos con las heces (no fuerzan el esfínter anal). 

En el exterior, por la temperatura y la labilidad del proglótide, se rompe y libera los huevos. Si en esa zona está el hospedador intermediario (por ejemplo, pastos de los que se alimenta el ganado bovino, ovino o porcino) puede ingerir los huevos, contaminándose. La cubierta radiada les protege del ácido estomacal, rompiéndose el embrióforo en el duodeno, surgiendo la oncosfera. La oncosfera, por medio de sus ganchos rompe la pared intestinal y busca un capilar, por el que viajará a tejidos subcutáneos, musculatura, ojo y cerebro. En estos órganos acontece el proceso de vesiculización: el parásito entra en estas células, transformándose en un cisticerco (Cysticercus cellulosae, o C. racemosus (en el cerebro)). Cuando el hospedador definitivo ingiere la carne contaminada, este pasa por el estómago sin problemas y llega al duodeno, donde se evagina el escólex del cisticerco, rompiéndose la vesícula que lo protegía, y se fija a la mucosa intestinal. Allí crece, en un proceso conocido como estrobilización, donde llega a su forma definitiva de adulto, pudiéndose cerrar el ciclo.

En esta especie existe la posibilidad de que el sr humano se comporte como hospedador intermediario, esto es, se infecte con el huevo y desarrolle cisticercos. Existen dos vías para ello: infestación heterógena (ingiera huevos en agua o alimentos); o autógena (exógena, vía ano-mano-boca; o por antipersitaltismo, que lleve a los anillos al duodeno y allí emerjan). 



















Taenia asiatica

Descubierta en 1993, presenta la misma morfología que T. saginatus, aunque presenta un ciclo parecido al de T. solium. El hombre se infecta mediante la ingestión de cisticercos de esta especie presentes en el hígado de cerdo (hospedador intermediario). Se distribuye fundamentalmente por las costas orientales de Asia. Dado que solo es distinguible a nivel molecular, no se sabe mayor información de este helminto.

Control:
• Lavado correcto de ensaladas.
• A nivel del hospedador definitivo: control de excretas y diagnóstico y tratamiento de hospedadores.
• Con respecto a hospedadores intermediarios: congelado y/o cocinado de la carne y control veterinario de la misma.

Teniosis

La teniosis es la enfermedad provocada por la fase adulta de T. saginata, y T. solium. Su periodo de incubación es de 2 meses, su periodo prepatente de 2 a 3 meses, y el patente, de 25 años. Este parásito lleva a cabo las siguientes acciones:
• Traumática: debido al rostelo (ventosas y ganchos), que rompe la submucosa y provoca una inflamación local.
• Tóxica: llevado a cabo por subproductos del metabolismo del parásito, que provocan prurito anal.

Los síntomas suelen ser muy leves, pudiéndose incluso dudar se estos no están influenciados por el nerviosismo del paciente al conocer al parásito. Estos son: dolores epigástricos, molestias abdominales, vómitos, diarreas, pérdida de apetito, pérdida de peso y prurito anal.



Cisticercosis

Causada por la forma de cisticerco de Taenia solium. Su periodo de incubación depende del órgano afectado, siendo antes en tejidos cerebrales, ojo, submucosa y, por último, músculo. Su periodo de incubación es de 2 a 3 meses, su periodo prepatente de 2 meses y el patente, 2 años. Las acciones que realiza son:
• Mecánica: de gran importancia en el cerebro, al bloquear ciertas regiones.
• Tóxica: sobre todo es importante en el músculo, al producir inflamación y calcificación.

Los síntomas dependen del tipo de cisticercosis:
• Cerebral: dependiente de la localización: cefalea, convulsiones, cambios de personalidad, delirios de grandeza, pérdida de visión, crisis epileptiformes, etc.
• Ocular: molestias oculares, pérdida de visión.
• Musculares: miosistis.

Diagnóstico

• Teniosis: tan sólo se puede recurrir al diagnóstico etiológico por coprología, ya sea visualizando los anillos grávidos, clasificando la especie según el número de ramas uterinas; o visualizando los huevos, sin posibilidad de clasificción dentro de los Ténidos.
• Cisticercosis: Mediante inmunoensayo (HAI, ELISA); o por técnicas de imagen (detección del cisitcerco (etiológico y clínico)): Rayos X, TAC, Escáner.
\newpage
\subsection{\textit{Hymenolepididae}}
Morfología

La familia Hymenolepididae es una familia de tenias cosmopolitas muy frecuente en niños, pero no tanto en adultos, que suelen ser inmunes. De esta familia, presentan repercusión sanitaria Hymenolepis nana e H. diminuta. Ambos son cestodos pequeños, con las siguientes características:
• Hymenolepis nana: de 0.7 a 10 cm, el adulto presenta un escólex globoso con rostelo armado (una corona, en forma de horquilla) y retráctil, un cuello muy marcado y 4 ventosas. Los anillos son más anchos que largos, con un solo aparato genital masculino y femenino. El poro genital es lateral, que se continúa con una vagina y un útero (saciforme), y sólo tres testículos.
El huevo es completo sin embrióforo radiado, en forma de limón con de 4 a 8 filamentos polares.
Presenta una forma de cisticercoide: una forma sólida (constituye la cola, situada en el extremo posterior) y una forma vesiculosa (estructura globosa donde se halla invaginado el escólex del adulto).
• Hymenolepis diminuta: de 20 a 50 cm, el adulto presenta un escólex globoso con rostelo inerme y retráctil, un cuello muy marcado y 4 ventosas. Los anillos son más anchos que largos, con un solo aparato genital masculino y femenino. El poro genital es lateral, que se continúa con una vagina y un útero (saciforme), y sólo tres testículos.
El huevo es completo sin embrióforo radiado, esférico y sin filamentos polares. Es mayor que H. nana.
Presenta una forma de cisticercoide: una forma sólida (constituye la cola, situada en el extremo posterior) y una forma vesiculosa (estructura globosa donde se halla invaginado el escólex del adulto).
Ciclo biológico

Se trata de un parásito heteroxeno con un hospedador intermediario, los redores, y ocasionalmente el ser humano (accidental, actúa como hospedador definitivo o intermediario). De la siguiente forma, presenta dos ciclos: indirecto y directo.

• Ciclo directo: sólo presente en H. nana, en el hospedador definitivo; acontece de la siguiente manera. Los huevos eliminados por el adulto pueden generar una autoinfección exógena (vía ano-mano-boca o vehiculizados en agua o verduras contaminadas) o endógena (huevos que eclosionan en el intestino). Fuera como fuere, entre las vellosidades intestinales se libera la oncosfera y acontece la vesicularización, formando, en este caso, un cisticercoide sin cola (tardando 4 días). Una vez completado el proceso, el cisticercoide se libera al lumen, donde se evagina, asciende hacia el duodeno y comienza la estrobilización. 

• Ciclo indirecto: presente en ambas especies, necesita de un hospedador intermediario y uno definitivo. El proceso es el siguiente:
Tras la fecundación de los proglótides maduros, se liberan los anillos grávidos, que al ser tan lábiles, se rompen antes de salir, siendo únicamente los huevos vehiculizados al exterior. Allí, estos huevos esperan al hospedador intermediario, que habrá de ingerirles. Lo hacen las fases larvarias de una pulga o de un colóptero, continuando el ciclo, por ser hospedadores intermediarios. Es necesario que el aparato bucal sea masticador y no picador para poder ser introducidos al intestino de estos animales.

En el interior de estos hospedadores intermediarios (situados en el ambiente del roedor, siempre cercano a depósitos de grano), la oncosfera, rompe las cubiertas que le protege, abandona el intestino de estos insectos y se establece en el hemocele de estos. Allí se transforman en cisticercoides y permanecen al encuentro con el hospedador definitivo.

La entrada al hospedador definitivo se produce cuando un roedor o el ser humano ingieren, de forma accidental, a estos insectos. En el intestino de estos hospedadores definitivos acontece la salida del cisticerocide, la evaginación del escólex y, una vez fijado a la pared intestinal, la estobilización 


Control:
• A nivel de hospedador intermediario, desinsectación (normalmente zonas próximas a graneros, cuyos trabajadores son población de riesgo)
• A nivel de hospedador definitivo, desratización y diagnóstico y tratamiento de hospedadores, así como una correcta higiene personal.

Hymenolepiosis

El periodo de incubación de la hymenolepiosis es de 1 a 4 semanas, el periodo prepatente de 2 semanas, y el patente, de 2 a 4 semanas. El parásito lleva a cabo las siguientes acciones:
• Traumática: generada por la oncosfera, cuyos ganchos provocan, sobre la mucosa intestinal, una reacción inflamatoria local, pero el movimiento de los adultos (no se sitúan siempre en el mismo lugar), que no poseen ganchos pero si ventosas, extiende por varios sitios esa inflamación.
• Tóxica: provocada por los metabolitos expulsados por este gusano platelminto.

La patogenia de este parásito consiste en el aplanamiento de las vellosidades por los cisticercoides (autoinfestación) y una enteritis superficial sin llegar a generar úlceras en la mucosa si existe una gran cantidad de parásitos. Esta parasitosis puede cursar de dos formas distintas:
• En adultos suele ser asintomática.
• En niños, por el ciclo de autoinfección, suelen ser masivas, mostrando: malnutrición, dolor abdominal, diarrea por aumento del peristaltismo, prurito anal y nasal (acción tóxica), nerviosismo (genera meteorismo, pérdida de peso y crisis epileptiformes).
Diagnóstico

• Etiológico: mediante coprología (visibles los huevos en heces por gran cantidad (infestaciones masivas) y el aspecto característico de los huevos.

\newpage
\subsection{\textit{Diphilium caninum}}
Morfología

Cestodo intestinal del perro y el ser humano (ocasionalmente). Llega a medir de 20 a 80 centímetros y tiene un diámetro de, en torno a 3 ó 5 mm. Presenta las siguientes fases en su ciclo biológico:
• Adulto: presenta un escólex globoso con 4 ventosas de gran tamaño y un rostelo piramidal y armado con de 3 a 7 coronas de ganchos en espinal de rosal (triangulares). A esta región le sigue el cuello y el estróbilo. En esta última, el estróbilo, presenta anillos maduros (en forma de tonel, presenta una simetría bilateral: tiene dos poros genitales en sendos lados, apatri de los cuales se localiza una vagina, un ovario y una glándula vitelógena postovárica. Junto al poro genital surge la bolsa del cirro, que se continúa con el vaso deferente, que se ramifica para ir a los 150 a 300 testículos presentes) y grávidos (en forma de tonel, presenta un útero muy desarrollado que se divide en cápsulas ovíferas en cuyo interior se hallan presentes los huevos).
• Huevo: presenta el huevo típico de los ténidos: un embrión hexacanto en su interior (tres pares de ganchos) y un embrióforo no radiado. Es completo, presentando dos cubiertas.
• Cisticercoide: presenta una forma sólida (constituye la cola, situada en el extremo posterior) y una forma vesiculosa (estructura globosa donde se halla invaginado el escólex del adulto).

Ciclo biológico

El ciclo comienza tras la autofecundación de los proglótides maduros, formándose los anillos grávidos, que son expulsados en grupos de 5 a 6 de forma activa (fuerzan el esfínter anal). En el exterior, los anillos se rompen y se liberan las cápsulas ovíferas, que son capaces de resistir hasta 25 días. Cuando una de estas es ingerida por una fase larvaria de una pulga (Ctenocephallides canis, Ctenocephallides felis) o un piojo masticador adulto (Trichodectes canis) del perro, continua el ciclo, por ser hospedadores intermediarios. Es necesario que el aparato bucal sea masticador y no picador para poder ser introducidos al intestino de estos animales.

En el interior de estos hospedadores intermediarios (situados en el ambiente del perro), la oncosfera, rompe las cubiertas que le protege, abandona el intestino de estos insectos y se establece en el hemocele de estos. Allí se transforman en cisticercoides (suele haber de 10 a 60 de estos) y permanecen al encuentro con el hospedador definitivo.

La entrada al hospedador definitivo se produce cuando el perro o el ser humano ingieren, de forma accidental, a estos insectos. En el intestino de estos hospedadores definitivos acontece la salida del cisticerocide, la evaginación del escólex y, una vez fijado a la pared intestinal, la estobilización 

Control:
• A nivel de hospedador intermediario: desinsectación de perro y entorno.
• A nivel de hospedador definitivo: diagnóstico y tratamiento de individuos parasitados.

Dypilidiosis

La dypilidiosis es una enfermedad que en adultos suele cursar de forma asintomática, así como ser especialmente poco frecuente; pero en niños, y debido a, principalmente, la acción tóxica del parásito, generada por la eliminación de metabolitos tóxicos para el organismo. Los síntomas son: anorexia, diarrea, pérdida de peso, prurito anal, nerviosismo y crisis epileptiformes.

Diagnóstico

• Etiológico: mediante coprología (visionado de huevos, cápsulas ovíferas o anillos en heces) u observación de cápsulas ovíferas en camas de gatos o entornos a animales de compañía (textura similar a arena blanquecina).
\newpage
\subsection{\textit{Echinococcus granulosus}}
Morfología

Gusano cestodo, parasito cosmopolita eurixeno. Presenta una forma adulta, presente en el intestino del perro (hospedador definitivo), y unas fases larvarias que están en los distintos hospedadores intermediarios.
• Adulto: de pequeño tamaño (hasta 7 mm), su cuerpo está formado por un escólex con 4 ventosas y un rostelo piriforme con una doble corona de ganchos en forma de puñal; un estróbilo de 2 a 3 anillos (inmaduro, maduro y grávido). El anillo maduro posee un ovario en forma de riñón, glándula vitelógena postovárica, entre 25 a 80 testículos y un poro genital en la mitad posterior. Los anillos grávidos tienen un útero longitudinal con cortas ramificaciones.
• Huevos: igual que el de otros ténidos: esférico, sin opérculo ni espolón, presentan una doble corona radiada (embrióforo) con un embrión (oncosfera) completamente desarrollado, con tres pares de ganchos (hexacanto). Fase de eliminación en el hospedador definitivo.
• Protoescólex: fase larvaria presente en el hospedador intermediario, si se halla invaginado tiene forma de grano de café con los ganchos muy marcados. Cuando se evagina, tiene la forma del escólex del adulto.

Otra formación que genera el parásito es el quiste hidatídico: En el órgano donde se asiente la oncosfera formará una cápsula formada por una membrana laminar (externa) y una germinativa (interna), de la cual saldrán las cápsulas prolígeras, con los protoescolex en su interior. En el interior de la capsula están los protoescólex libres, vesículas hijas endógenas viables o no, y restos degenerados de otras estructuras. Por acción del sistema inmune, se forma una membrana adventicia, formada por el parénquima del órgano que infectan, friable de las otras dos membranas. Estos pueden ser infértiles (sin protescólex), fértiles (con protoescólex) o hiperfértiles (con muchas vesículas hijas); o degenerado (calcificado).

Ciclo biológico

El ciclo biológico comienza en el intestino del perro, el hospedador definitivo, con la salida de los huevos del anillo grávido, saliendo los huevos, ya embrionados, por las heces. Estos contaminan el pasto donde pacen los hospedadores intermediarios: ovejas, cerdos, reses,…

En el intestino de los hospedadores intermediarios, el embrióforo se rompe y surge la oncosfera, que rompe la pared intestinal en su búsqueda de un capilar, donde viajará a distintos órganos, preferentemente hígado y pulmones, pero pudiendo llegar a corazón, interior de huesos y cerebro. Allí formarán el quiste hidatídico

El ciclo es continúa cuando el hospedador definitivo ingiere las vísceras del hospedador intermediario contaminadas con estos quistes hidatídicos. El protoescólex, que se halla invaginado, aguanta el paso por el estómago y se evagina en el duodeno del perro. Aquí, se ancla con sus ventosas y su rostelo a la pared intestinal. Allí acontece el proceso de estrobilización y, una vez formados los anillos, se produce la fecundación, cerrándose el ciclo.
En este caso, el humano es un hospedador accidental, actuando como hospedador intermediario, contaminándose por huevos presentes en comida o por presentes en el pelaje del perro que acaban por llegar a la zona oral.

Control:
• Tratamiento de hospedadores definitivos (perros)
• Control de carnes que sirvan como fuente de alimentación al perro.

Hidatidosis

Las acciones que lleva a cabo el parásito son:
• Traumática: llevada a cabo por la oncosfera, esta irrita al órgano en el que está, formando una membrana adventicia sobre el parásito.
• Mecánica: la lleva a cabo el quiste generando una presión contra el órgano donde se ubica. Resulta más severa en el cerebro.
• Tóxica: por salidas de líquido hidatídico (que puede dar lugar a quistes secundarios), que lleva algunos metabolitos tóxicos para el hospedador. Dependiendo de la menor o mayor cantidad de líquido puede ser local (necrosis del tejido adyacente) o generalizada, que puede producir un shock anafiláctico y, con ello, la muerte.

Los síntomas suelen ser pobres, sino inexistentes, aunque depende del órgano que afecte.

Diagnóstico

• Etiológico: se realiza mediante visualización del quiste por técnicas de imagen (Rayos X, Escáner, Ecografía). Nunca se harán biopsias por riesgo a la rotura y diseminación de protoescólex.
• Inmunológico: se realiza mediante intradermoreacción de Casoni (desaconsejada, se trata de ver una reacción inmune por inoculación parenteral de antígeno del parásito; produciéndose inmunidad contra este una vez realizada la prueba), HAI, ELISA, IFI o Inmunoelectroforesis, mediante la visualización del arco 5(tiene reacciones cruzadas con E. multilocularis).
\newpage
Echinococcus multilocularis
Morfología

Gusano cestodo, parasito restringido al Hemisferio Norte (Norte de Asia, Canadá y parte de EEUU, Rusia, Alemania, Suiza, Austria; Grecia y Turquía) eurixeno. Presenta una forma adulta, presente en el intestino del perro (hospedador definitivo), y unas fases larvarias que están en los distintos hospedadores intermediarios.
• Adulto: de pequeño tamaño (hasta 5 mm), su cuerpo está formado por un escólex con 4 ventosas y un rostelo piriforme con una doble corona de ganchos en forma de puñal; un estróbilo de 4 a 5 anillos (inmaduro, 2 maduros y 2 grávidos). El anillo maduro posee un ovario en forma de bilobulado, glándula vitelógena postovárica, en torno a 20 testículos y un poro genital en la mitad anterior. Los anillos grávidos tienen un útero sacular con cortas ramificaciones.
• Huevos: igual que el de otros ténidos: esférico, sin opérculo ni espolón, presentan una doble corona radiada (embrióforo) con un embrión (oncosfera) completamente desarrollado, con tres pares de ganchos (hexacanto). Fase de eliminación en el hospedador definitivo.

• Protoescólex: fase larvaria presente en el hospedador intermediario, si se halla invaginado tiene forma de grano de café con los ganchos muy marcados. Cuando se evagina, tiene la forma del escólex del adulto.

Otra forma, sólo presente en el hospedador intermediario, es la del quiste hidatídico alveolar: forma una red de túbulos que se infiltra en el órgano. Presenta una membrana adventicia (externa) generada por el hospedador a partir del parénquima del órgano, una adventicia y otra germinativa más interna, formadas por el parásito, que forman las estructuras de protoescólex y vesículas hijas, que en el caso del ser humano, son infértiles.

Ciclo biológico

El ciclo biológico comienza en el intestino del zorro o del gato, hospedadores definitivos, con la salida de los huevos del anillo grávido, saliendo los huevos, ya embrionados, por las heces. Estos contaminan el pasto donde pacen los hospedadores intermediarios: roedores.

En el intestino de los hospedadores intermediarios, el embrióforo se rompe y surge la oncosfera, que rompe la pared intestinal en su búsqueda de un capilar, donde viajará a distintos órganos, preferentemente hígado y pulmones, pero pudiendo llegar a corazón, interior de huesos y cerebro. Allí formarán el quiste hidatídico

El ciclo es continúa cuando el hospedador definitivo ingiere las vísceras del hospedador intermediario contaminadas con estos quistes hidatídicos. El protoescólex, que se halla invaginado, aguanta el paso por el estómago y se evagina en el duodeno del zorro o del gato. Aquí, se ancla con sus ventosas y su rostelo a la pared intestinal. Allí acontece el proceso de estrobilización y, una vez formados los anillos, se produce la fecundación, cerrándose el ciclo.
En este caso, el humano es un hospedador accidental, actuando como hospedador intermediario, pero en el que las hidátides son infértiles, contaminándose por huevos presentes en comida (fresas silvestres).

Control:
• En el caso humano, no comer frutos silvestres cercanos al suelo (especialmente fresas silvestres)

Hidatidosis alveolar

De pronóstico grave, llega a ser mortal. Similar a la hidatidosis unilocular, suele formar hidátides en el hígado, pero también en el pulmón y en el cerebro. Esta enfermedad cursa con: hepatomegalia, hígado duro e irregualr a la palpación, insuficiencia hepática, ictericia, cirrosis y, en pulmón y cerebro, un crecimiento que recuerda a metástasis de algunos tumores.


Diagnóstico

El diagnóstico suele ser complejo. La forma del quiste hidatídico multilocular puede recordar a la de un tumor, no recomendándose la biopsia ante el riesgo de diseminación. Las técnicas inmunológicas más sensibles son IF, ELISA y Inmunoelectroforesis, mediante la visualización del arco 5 (tiene reacciones cruzadas con E. granulosus), pero pueden dar falsos positivos en neoplasias o de otras hidatidosis. 
\newpage
\subsection{\textit{Diphylobotium latum}}
Morfología

Descrito ya en el papiro de Ebers, se le conoce como la “tenia ancha del pez”. Suele distribuirse en aguas frías: Mar Baltico, Mar del Norte, lagos de Suiza, Mar de Barents, Mar Jónico y Tirreno y Costas del Pacífico Norte.

Llega a medir hasta 20 metros en su forma adulta, siendo lo más normal que se sitúe entre los 30 y 8 metros; se trata de un cestodo que llega a poseer hasta 3000 o 4000 proglótides. Se pueden distinguir las siguientes fases vitales en su ciclo vital:
• Adulto: presenta un escólex (pequeño, piriforme con un par de botrios laterales, sin rostelo armado); un cuello; y un estróbilo.
Los anillos maduros poseen un poro genital único y ventral y anterior al tocostoma (orificio por el que salen los huevos). El útero es central y tiene forma de roseta, mientras que el útero es bilobulado. Posee un gran número de testículos y de folículos vitelógenos, que se distribuyen por todo el anillo.
El anillo grávido  presenta un útero muy desarrollado 
• Coracidio: surgido del huevo en un ambiente acuoso, consiste en una estructura similar al embrión de un tenido (esférico, con seis pares de ganchos (embrión hexacanto)) con una capa externa de células ciliadas que le permiten nadar en ese medio.
• Larva procercoide: de aspecto alargado, presenta en su parte anterior unas glándulas de penetración y los ganchos en su parte más posterior, en un cuerpo vesiculoso surgido por estrangulamiento de esta región denominado cercómero.
• Larva plerocercoide: fase presente en el pez, es similar al adulto, presenta los botrios y una pseudometamerización en todo el cuerpo.
• Huevo: de color marrón oscuro, operculados y de extremos redondeados, sus medidas son de 71x51 $\mu$m.


Ciclo biológico

Se trata de un parásito heteroxeno y eurixeno, que sigue un ciclo con dos hospedadores intermedios y una amplia gama de hospedadores definitivos (mamíferos ictiófagos).

El ciclo comineza con la autofecundación del parásito en el intestino del hospedador definitivo. Los huevos salen por el tocostoma del anillo grávido y son vehiculizados por las heces al exterior, estando en el momento de la salida en el estado de mórula (no están embrionados). Tras caer en un medio acuático, en unos 11 a 15 días en temperaturas de 8 a 30ºC, la forma de coracidio surge del huevo. Esta forma nada en la superficie hasta que es ingerida por un copépodo de los géneros: Diaptomus, Metocyclops, Cyclops. En el interior de este se forma la larva procercoide, situándose en su intestino.

Cuando un pez de la familia de los salmónidos (lucio, perca, salmón,…) ingiere a un copépodo parasitado, se libera el procercoide en el intestino del pez y abandona este aparato en dirección a la musculatura del pez. Allí, este se transforma en plerocercoide. En ese momento puede acontecer que sea ingerido por un hospedador paraténico o por el definitivo. Cuando el hospedador definitivo ingiera la carne (del hospedador intermediario o paraténico) contaminada con esas larvas plerocercoides, en el duodeno acontecerá la estrobilización y desarrollo del parásito, cerrándose así el ciclo.
Control:
• Evitar la ingesta de pescado poco cocinado o no congelado.
• Diagnóstico y tratamiento de individuos no parasitados.
• Control de excretas.
Botriocefalosis

El periodo de incubación de la enfermedad es de 21 días; el periodo prepatente es de 21 a 24 días; y el patente, de hasta 10 años. Las principales acciones de este parásito son:
• Mecánica: al situarse en el duodeno, si la carga parasitaria es alta puede llegar a obstruir el intestino.
• Expoliadora: este parásito detenta hasta el 50% de la vitamina B12 que pasa por el intestino, impidiendo su absorción.

La mayoría de las botriocefalosis suelen ser asintomáticas (portadores sanos), pero los hospedadores sintomáticos presentan el siguiente cuadro de síntomas: diarrea, pérdida de peso y dolor abdominal; cefaleas; reacciones alérgicas; obstrucción intestinal si la carga parasitaria es alta; colecistitis o colanguitits por la migración de proglótides a regiones anteriores del intestino, donde pueden germinar; y déficit de vitamina B12, que se manifiesta con anemia y neuropatía periférica y lesiones en el SNC.

Diagnóstico

• Clínico: la presencia de anemia megaloblástica, déficit de vitamina B12 y una eosinofilia en sangre periférica (5 a 10% de los pacientes), son indicativos de una parasitemia por este gusano.
• Etiológico: mediante coprología (el hallazgo de huevos en heces se puede hacer a partir de las 6 semanas) o mediante técnicas de diagnóstico molecular diferencial entre especies (PCR).
\newpage
\section{Clase \textit{Trematoda}}
\subsection{\textit{Fasciola hepatica}}
Morfología y distribución

De gran tamaño, cosmopolita, Fasciola hepática es el agente etiológico de la fasciolosis. El adulto presenta una ventosa oral prominente en el cono apical. Ciegos intestinales ramificados, hermafroditas: su aparato reproductor lo forma un ovario ramificado, dos testículos posteriores a estos y unas glándulas vitelógenas extendidas por el margen lateral y posterior del cuerpo.

Los huevos son grandes, ovalados, simétricos y con una cáscara fina, operculado en un extremo. Las cercarias presentan un extremo anterior cónico y una cola natatoria.

Ciclo biológico

El adulto suele vivir en las vías biliares, en su parte proximal, del hospedador definitivo (cerdos, ovejas o el ser humano). En ellas se alimenta de bilis, se autofecunda y pone los huevos, que caen por el colédoco al duodeno y salen vehiculizados por las heces. Esos huevos salen no embrionados, necesitando un medio húmedo (charcos,…) para terminar de formarse. Tras 10 días, a 22ºC y por la luz, el huevo se rompe saliendo el miracidio por el opérculo, necesitando encontrar al hospedador intermediario (caracol de la familia Lymnaeidae, como Lymnaea trunculata, presente en España).El miracidio entra al hospedador intermediario por los tentáculos o por el pie. En el punto de penetración forma un esporoquiste que genera las redias madres, que buscan el hepatopáncreas, donde se forman las redias hijas, que darán lugar a las cercarias, que abandonan el caracol saliendo con el moco.

Las cercarias nadaran hasta zonas con vegetación, donde se enquistan, transformándose en metacercarias. Si son ingeridas por el hospedador definitivo, las metacercarias pasan por el estómago, liberando al trematodo joven en el duodeno. Allí rompen la mucosa dirigiéndose al peritoneo, buscando el hígado e introduciéndose en los conductos biliares, cerrándose el ciclo. No obstante, puede perderse en su viaje al hígado, encapsulándose en regiones ectópicas (pulmón,…) y calcificándose.

Control:
• Higiene alimentaria: limpieza de alimentos (especialmente berros) para ensaladas.
• Tratamiento de individuos parasitados.
• Desplazamiento del hospedador intermediario mediante molusquicidas, competencia con otras especies,…

Fasciolosis

El periodo de incubación es de 3 a 12 semanas, el prepatente de 3 meses y el patente de hasta 20 años. El parásito lleva a cabo las siguientes acciones en el hospedador:
• Traumática: se da en el hígado y en los conductos biliares por el tegumento espinoso y las ventosas del parásito y el movimiento de este en el parénquima hepático, que llevan a una reacción inflamatoria acompañada de eosinofília con una fibrosis portal e hiperplasia de los conductos biliares.
• Tóxica: por la eliminación de ciertos metabolitos que causan daño en el hospedador (prolina)
• Mecánica: obstructiva, por el bloqueo de los conductos biliares que se produce por la presencia en estos del parásito y de la hiperplasia de estos por la presencia del parásito.

Esta enfermedad se caracteriza por unos síntomas muy generales (que suelen acompañar a muchas enfermedades) y algunos más específicos hepáticos por ser este el órgano más afectado por la parasitemia. Son: dolor abdominal intenso (en las primeras semanas), cefalea intensa, vómitos, diarrea, anemia, escalofríos y gran eosinofilia, urticaria y edema submandibular, hepatomegalia dolorosa al tacto, ictericia y abscesos en lugares ectópicos.

Diagnóstico

• Clínico: por la presencia de un bloqueo hepático asociado a una hepatomegalia dolorosa, síndrome febril eosinofílico y consumo de berros.
• Etiológico: mediante la búsqueda de huevos en coprologías. Para evitar falsos positivos, se recomienda no comer hígado en los tres días anteriores a la prueba. 
• Indirecto: muy útil en el periodo prepatente, se usa HAI, ELISA o IFI.
\newpage
Morfología y distribución

La opisthorchiidosis es una parasitosis provocada por gusanos planos (platelmintos) de la familia Ophisthorchidae. En concreto, los géneros más representativos que tienen una significancia en el campo clínico son Clonorchis (C. sinensis, o “duela de China”, de Japón, Tailandia, Korea y Vietnam) y Ophisthorchis (O. felineus y O. viverrini, el primero presente en el antiguo Pacto de Varsovia; y el segundo originario de Indochina).
Ambos dos son gusanos pequeños, con forma de bala o huso (el adulto), y un huevo no distinguible.
• Adulto: todas las especies son iguales, salvo en los testículos (en Clonorchis son ramificados y en Ophisthorchis son lobulados). Posee una ventosa apical que se continúa con la faringe, el esófago y se bifurca en dos ciegos no ramificados. A los lados, en situación lateral, de los ciegos intestinales se hallan las glándulas vitelógenas. En situación medial y posterior a la bifurcación de los ciegos se halla la ventosa ventral y, posterior a ella, el útero, los ovarios y los testículos (en la posición más terminal.
• Huevo: forma esférica con un opérculo con estructuras anexas (hombros del opérculo) que lo refuerzan y un tubérculo en situación opuesta.

Estos géneros se encuentran sobre todo en Asia (Rusia Occidental y Kazajistán, China, Corea, Indochina y Filipinas). Existen unos 19 millones de casos en Asia, y casos en Europa derivados del turismo a esas zonas. En Tailandia hay hasta 6 millones de casos. En España se han encontrado fases larvarias en algunos peces.

Ciclo biológico

El parasito, en su forma adulta, vive en los conductos biliares del ser humano, siendo su hospedador definitivo. Allí, tras alimentarse de bilis, se autofecunda y realiza la puesta de huevos, que van por los conductos biliares hasta el duodeno, y de ahí, salen con las heces al exterior. Esos huevos ya están embrionados, necesitando, para seguir el ciclo, encontrar con rapidez al siguiente hospedador, un caracol del género Bithynia.

Los huevos, que llevan al miracidio, se rompen en el intestino, liberando a esa fase larvaria (el miracidio). Del tubo digestivo migrará hacia el hepatopáncreas, pero, en el momento de salir, formará un esporoquiste del que saldrán las redias madres que formarán en el hepatopáncreas a las redias hijas que se transformaran a su vez en cercarias (de parte anterior alargada y con dos formaciones oscuras en forma de ojos (oftalmocercarias) y una cola con aleta natatoria). Estas cercarias dejan el caracol saliendo junto con el moco por su superficie plantar. Para continuar el ciclo, han de encontrar al hospedador intermediario siguiente, un pez de la familia de los ciprínidos o de los salmónidos, en las siguientes 48 horas, ayudándose para un mejor contacto de un geotropismo negativo y un fototropismo positivo.

Una vez han contactado con el pez, las cercarias se introducen bajo sus escamas y se enquistan en su fase de metacercaria en la musculatura del pez. Cuando un humano ingiere la carne del pez contaminada con metacercarias, estas aguantan los jugos gástricos y en el duodeno, liberan al trematodo joven. Este busca el colédoco y se dispone en los conductos biliares, cerrando el ciclo.

Control:
• Cocinado de pescados (a más de 50ªC) o congelación (-20ºC) para la eliminación de metacercarias.
• Tratamiento de parasitados.
• Desplazamiento de hospedadores intermediarios (molusquicidas, competencia con otras especies de caracoles,…)
• Control de aguas de letrinas.
Opisthorchiisodis

El periodo de incubación de esta parasitosis es de 2 semanas; un periodo prepatente de 2 a 4 semanas y un periodo patente de 20 años. La gravedad de la parasitosis dependerá del número de parásitos (menos de 100 suele cursar asintomática) y de la duración de la misma (relacionándose con la posibilidad de sufrir hepatitis: cuanto mayor es el tiempo que existe un parasitismo, mayor es esa posibilidad).

Las acciones que realiza este parásito son las siguientes:
• Traumática: el movimiento del parásito en los conductos biliares provoca una reacción inflamatoria eosinofílica. El engrosamiento de los conductos biliares (inflamación) dará lugar a una hiperplasia adenomatosa y fibrosa en conductos pancreáticos (por localización ectópica del parásito, lo que lleva a una pancreatitis) y biliares. 
• Mecánica: de bloqueo, debido a la presencia del parasito y la inflamación de los conductos biliares y pancreáticos, que reducen el lumen y forman cálculos biliares y daños en el páncreas. 
• Tóxica: el parásito, producto de su metabolismo, desecha subproductos que resultan tóxicos para el hospedador.
• Vehiculadora: en su paso del duodeno a los conductos biliares y pancreáticos puede llevar consigo ciertas bacterias intestinales que generen enfermedades por su localización ectópica.

La enfermedad posee dos fases, en las que suele haber una gran eosinofilia (40%), y dolor hepático a la palpación. Fases:
1. Aguda: los síntomas suelen ser fiebre, diarrea, dolor epigástrico, anorexia, cirrosis portal, hepatomegalia, ictericia, esplenomegalia. Si los huevos alcanzan localizaciones ectópicas pueden causar alteraciones nerviosas, sensoriales y cardiacas. 
2. Crónica: se acompaña de apetito irregular, distensión abdominal, diarrea y edema.

La opisthorchiidosis es también un factor de riesgo ante carcinomas hepáticos.

Diagnóstico

• Etiológico: mediante búsqueda de huevos de estas especies en heces (coprología).
\newpage
\subsection{\textit{Fasciolopsis buski}}
Morfología

Gusano trematodo de Sureste asiático y Extremo Oriente, con una prevalencia de 10 millones de casos anuales y varios reservorios (perro, cerdo, gatos, conejos,…) y al hombre como hospedador definitivo. Es uno trematodo “carnoso”, por su gran grosor.

El adulto mide hasta 75 mm de longitud, 20 mm de ancho y 5 mm de grosor. Tiene el tegumento espinoso y dos ventosas, oral y ventral, esta última siendo la más grande. En la ventosa ventral se abre la boca, que se continúa con una faringe musculosa y dos ciegos intestinales longitudinales no ramificados. En los laterales se hallan las glándulas vitelógenas, que drenan a un viteloducto que acaba en el ootipo, adonde llegan óvulo y espermatozoides. Superior al ootipo es útero, que lleva los huevos al exterior por el poro genital. Los testículos se hallan muy ramificados y dispuestos postovaricamente.

Los huevos son ovalados, grandes y operculados.

Ciclo biológico

El ciclo comienza tras la autofecundación en el intestino del hospedador definitivo. En las heces del individuo son vehiculizados al exterior gran cantidad de huevos no embrionados, que tras 3 a 7 semanas en un medio acuático, forman el miracidio. Este ha de encontrar en un máximo de 48 horas al siguiente hopeddor, un caracol de los géneros Segmentia o Hippetius. Este penetra por el pie, atravesando el tegumento, momento en el que pierde el tegumento ciliado y se transforma en esporoquiste. De este surgirán las redias madres y las redias hijas, que se desarrollan en el hepatopáncreas, formando las cercarias, que salen del caracol por el moco que libera por su pie. Las cercarias nadaran hasta encontrar el pericarpo de castañas de agua, frutos de loto o raíces de bambú, donde se transformaran en metacercarias (pierden la cola y forman una cubierta resistente).

Cuando el hospedador definitivo ingiere esos frutos con el pericarpo contaminado con metacercarias, cierra el ciclo. Estas metacercarias se exquistarán en el duodeno, dad la acción de los jugos gástricos, la bilis y la temperatura, liberando al trematodo joven, que se autofecundará y dará lugar a más huevos.

Control:
• Desplazamiento del hospedador caracol mediante molusquicidas o competencia con otros caracoles.
• Saneamiento de aguas procedentes de letrinas.
• Diagnóstico y tratamiento de individuos parasitados.
• Lavado de castañas de agua, frutos de loto o raíces de bambú con hipoclorito sódico (3 gotas/1 l de agua)

Fasciolopsis

El periodo de incubación de la parasitosis es de 1 a 2 meses; el periodo prepatente es de 3 meses; y el patente, de 1 año. La acción patógena del parásito esta generada por los adultos fijados en la mucosa intestinal (duodeno), explicándose así la sintomatología:
• Acción traumática: ulceración inflamatoria del duodeno por el roce con el tegumento espinoso del parásito con posible formación de hemorragias: diarrea tóxica persistente y dolor abdominal.
• Acción mecánica obstructiva: alteran la secreción de jugos gástricos y secreción de moco. Gran número de ejemplares puede obstruir el intestino.
• Acción tóxica: sensibilización a metabolitos tóxicos del parásito: toxemia y alergia, con edema facial y en la pared abdominal.
A veces se acompaña de anorexia, náuseas y vómitos.

Diagnóstico

• Etiológico: el más concluyente (no es posible uno clínico y no hay estimulación antigénica), visionado de huevos en coprologías (sólo tras tres meses de parasitación).
\newpage
\subsection{\textit{Heterophydae}}
Morfología

Parásitos trematodos de Extremo Oriente, Delta del Nilo, Turquía, India, China Y Japón (Heterophyes heterophyes); y Siberia y Báltico (Metagonimus yokogawai). Miden en torno a 1.7 mm de largo y 0.4 mm de ancho. Los huevos de esta familia son los de los trematodos típicos, (operculados), si bien estos son más pequeños. Las especies más importantes son:
• Metagonimus yokogawai: el útero asciende hasta el poro genital (en el centro de la ventosa ventral, ligeramente inclinada y saliente por un lado). No tiene gonotilo. Las glándulas vitelógenas son posteriores y laterales. Presenta un tegumento espinoso en el tercio anterior del cuerpo. Presenta una faringe que se bifurca en dos ciegos.
• Heterophyes heterophyes: más pequeño, órganos reproductores en la parte posterior del cuerpo: testículos lobulados. El ovario tiene localización pretesticular, al igual que el ootipo, donde se forma el huevo y a partir de él, sale el utero. Este va al poro genital. Este poro genital se abre en el centro de la ventosa ventral, que se llama gonotilo.

Ciclo biológico

Este parásito tiene como hospedador definitivo al ser humano y a otros animales, como reservorios (gatos, perros, aves), siendo por ello un parásito heteroxeno.

El ciclo comienza tras la autofecundación en el intestino del hospedador definitivo: el adulto realiza la puesta de huevos (ya embrionados) que son vehiculizados con las heces hacia el exterior. Estos huevos han de ser ingeridos con rapidez por un caracol (géneros Pirinela, Cerithidia, Melania para H. hetreophyes; Semisulcospira para M. yokogawai). 

Una vez en el intestino del caracol, se libera el miracidio, que se dirigirá hacia el hepatopáncreas, pero, en el momento de salir, formará un esporoquiste del que saldrán las redias madres que formarán en el hepatopáncreas a las redias hijas que se transformaran a su vez en cercarias (de parte anterior alargada y con dos formaciones oscuras en forma de ojos (oftalmocercarias) y una cola con aleta natatoria). Estas cercarias dejan el caracol saliendo junto con el moco por su superficie plantar. Para continuar el ciclo, han de encontrar al hospedador intermediario siguiente, un pez (de agua dulce para M. yokogawai; marino para H. heterophyes).

Una vez contacta con el pez, la cercaria penetra por el tegumento, perdiendo la cola. La cercaria, en la musculatura del pez, se redondea y enquista, formando la metacercaria. Cuando el hospedador definitivo ingiere estas metacercarias, aguantan hasta llegar al duodeno, donde surge el trematodo joven, cerrando el ciclo.

Control:
• Cocinado o congelado de pescado (río o mar).
• Diagnóstico y tratamiento de personas parasitadas.
• Control de aguas residuales: saneamientos y letrinas.
• Eliminación del hospedador caracol mediante molusquicidas o competencia con otras especies.
Heterophydiosis

El periodo de incubación de esta enfermedad es de 1 a 3 semanas; el prepatente, de 5 a 10 días; y el patente, de 2 a 6 meses. La acción patógena de este parásito depende del número de adultos o huevos, del grado de penetración de estos parásitos en la pared intestinal y la reación del paciente. Las acciones que lleva a cabo en el hospedador son:
• Traumática: la pueden generar los adultos (inflaman la mucosa intestinal y generan la hipersecreción de moco, pudiendo, en casos de parasitosis masivas, generar perforaciones de las mucosas) o los huevos (pueden llegar a la sangre y generar lesiones en regiones ectópicas: corazón (reacción tisular en válvulas cardiacas) o SNC (trastornos neurológicos).

Los síntomas suelen ser diarrea crónica y dolor intestinal, pudiendo acompañarse de úlceras duodenales.

Diagnóstico

• Etiológico: mediante coprología, visionado de huevos, aunque pueden confundirse entre otros heterofídeos.
\newpage
Paragonimus westermani
Morfología

Trematodos de la familia Troglotrematidae, la especie Paragonimus westermani se distribuye por Extremo Oriente, India, Asia, África y Latinoamérica. Presenta dos formas: adulto (en el hospedador definitivo, presente en parejas con posibilidad de fecundación cruzada, encapsulados en nódulos de hasta 4 cm de diámetro) y huevo.

Tiene una forma ovoide, midiendo de 7.5 a 12 mm de longitud, 4 a 6 mm de anchura y hasta 5 mm de grosor. Con dos ventosas del mismo tamaño en región apical y ventral. Tiene unas glándulas vitelógenas muy desarrolladas que ocupan las regiones laterales. El ovario, lobulado, se halla ventrolateral, superior a los testículos, y en posición contraria al útero. El poro genital se encuentra tras la ventosa ventral. En el centro del cuerpo está la galndula excretora. Posee unos ciegos intestinales no ramificados y una faringe muy musculada.

Los huevos son ovalados, operculados, rojizos y con un engrosamiento en la cutícula opuesta al opérculo. 

Ciclo biológico

El ciclo comienza en el hospedador definitivo (ser humano y animales que se alimentan de crustáceos). En él, tras la autofecundación, se forman los huevos, que son eliminados en la expectoración (esputo) o por las heces, tras el tragado de esa expectoración. Una vez en el exterior, en un medio acuático, y tras largo tiempo (16 a 25 días, dependiendo de la temperatura) acontece el embrionado del huevo, surgiendo a posteriori el miracidio. Esta fase es temporal, teniendo que, rápidamente, encontrar al hospedador intermediario primero, un caracol de la familia Thiaridae: Semisulcospira, Potadoma, Brotia. En este penetra por el tegumento del pie.

En el punto de entrada del caracol, pierde la capa ciliada y se transforma en esporoquiste, que formará las redias madres de las que saldrán las redias hijas (que se desarrollan en el hepatopáncreas) con el moco, se expulsan las cercarias, surgidas de las redia hijas, que, por tener en este caso una cola pequeña (en botón) se les denomina microcercarias, o xifiliocercarias, por el estilete en su polo oral. Estas cercarias usaran a un cangrejo de agua dulce como segundo hospedador intermediario.

Las cercarias entran en este a través de las branquias, siendo vehiculizadas en el agua con el aspirado de agua. En las branquias, tras 6 semanas, forman una cubierta resistente, pasando a su forma de metacercaria. Si el cangrejo es ingerido por un hospedador definitivo, en el duodeno de este se rompe la metacercaria, liberando al trematodo joven, que perfora el duodeno hasta salir a la cavidad peritoneal de la que migrará al pulmón por sangre o perforando el diafragma, (los que se pierdan, formaran quistes en localizaciones ectópicas). Si es ingerido por un hospedador paraténico, el trematodo joven se enquistará en la musculatura, pudiendo resistir también a los ácidos estomacales, pudiendo ser también infectiva para el hospedador definitivo, en el que realizará el mismo proceso de migración.

Control:
• Diagnóstico y tratamiento de hospedadores definitivos
• Control de heces y esputos.
• Cocinado de la carne de cerdo (hospedador paraténico, se elimina al trematodo joven).
• Eliminación del hospedador intermediario caracol mediante molusquicidas o competencia con otros moluscos.
• Cocinado de crustáceos, no ingesta o manipulación de estos animales previo cocinado (los jugos de estos animales son capaces de llevar metacercarias).

Paragonimosis

El periodo de incubación es de 9 a 12 semanas; el periodo prepatente es de 10 a 12 semanas; y el patente, de hasta 20 años. En el organismo, el parasito lleva a cabo las siguientes acciones:
• Traumática: la realizan los adultos, genera inflamación en los pulmones, recubriéndolos, por parte del hospedador, de una capsula fibrosa. Los huevos suelen rodearse de neutrófilos y esosinófilos formando quistes o abcesos.
• Tóxica: los huevos degenrados o los adultos son capaces de generar metabolitos tóxicos que provocan urticaria, microbacesos o la formación de granulomas.

Mientras el parasito está en su fase invasiva (periodo de incubación), los hospedadores son asintomáticos. Una vez los adultos están en el pulmón se genera una reacción hística del hospedador que forma una cápsula fibrosa sobre estos parasitos y/o sus huevos, llevando a una fibrosis difusa generalizada o localizada en el pulmón. También es importante las localizaciones ectópicas del parásito, capaces de generar granulomas en otras regiones. La sintomatología asociada a esta enfermedad es:
• Pulmón: neumonía persistente, fiebre, accesos tuberculoides acompañados de tos  persistente, hemoptosis coincidente con dolor torácico opresivo, neumotórax, esputos de color amarillo, con sangre y puntos negros. Estos síntomas, asociados a la fibrosis, empeoran con el ejercicio físico.
• Abdomen: producida por la presencia de parásitos en la cavidad peritoneal: dolor abdominal, utricaria, diarreas con aparición de huevos.
• Cerebral: genrada por loaclizaciones ectópicas de huevos, dándose casos en Japón y Corea. Produce una epilepsia similar a la cisticercosis (niños menores a 15 años), y una meningitis, que, por la presión intracraneal genera: nauseas, vómitos, edema en el nervio óptico, dolor de cabeza, paralisis y coma.

Diagnostico

• Clinico: visibles los quistes nodulares mediante rayos X, TAC y Resonancia Magnética (estas dos últimas sobre todo son útiles en lesiones cerebrales). Así mismo, una gran esoinofilia (91%), leucocitosis y gran cantidad de IgE, y ciertas características del esputo (la gran cantidad de huevos) son también decisivas.

• Etiologico: búsqueda de huevos en heces, esputo o aspirado duodenal, junto con la visión de cristales de Charcot-Leyden.

• Indirecto: pueden existir reacciones cruzadas con Fasciola o Clonorchis. Se usan la técnica de ELISA, Inmunoblot, Contrainmunoelectroforesis, Fijación del complemento (muy sensible, pero muy larga) o intradermoreacción (poco aconsejable, es muy poco sensible y, una vez se ha entrado en contacto con el parásito, genera una reacción inmune).
\newpage
\subsection{\textit{Clonorchis sinensis}}
\newpage
\subsection{\textit{Schistosoma} spp}
La esquistosomosis es una enfermedad parasitaria causada por gusanos platelmintos de la clase Trematoda, siendo, los que parasitan al ser humano, cinco especies:
• Schistosoma mansoni: África y América del Sur (importada con el tráfico de esclavos).
• Schistosoma haematobium: Cuenca del Nilo y Próximo Oriente.
• Schistosoma japonicum: China, Japón y Filipinas
• Schistosoma intercalatum: África ecuatorial (de baja incidencia y restringido geográficamente, no muy prevalente)
• Schistosoma mekongi: Cuenca del Mekong: Laos, Camboya, Tailandia (de baja incidencia y restringido geográficamente, no muy prevalente)
La importancia de estas parasitosis reside en la prevalencia. Existen un total de 600 millones de personas expuestas, con unos 200 millones parasitadas, siendo 120 millones portadores sanos y 80 millones dan síntomas.

Morfología

Este género representa una excepción dentro de la familia. Estos parásitos viven en los plexos venosos del intestino o de la vejiga urinaria; son dioicos, con un gran dimorfismo sexual (la hembra es cilíndrica y el macho plano, pero forma un canal ginecóforo que alberga a la hembra durante la cópula); carecen de faringe; y presentan un aparato digestivo ciego y bifurcado con una unión cecal posterior a la ventosa ventral. Así mismo, carecen de fase de redia, teniendo una nueva fase larvaria en el hospedador definitivo, la esquistosómula.

• Huevo: ovalados, no operculados pero con un espolón lateral (S. mansoni y S. japonicum, este último poco promienente) o terminal (S. haematobium).
• Cercaria: o furcocercaria, es la fase infectiva para el hospedador definitivo. Forma libre que sale del caracol, posee una cola bifurcada natatoria que le permite buscar al siguiente hospedador, en su parte terminal. Su parte anterior lo ocupa el cuerpo de la cercaria, que posee en su extremo apical un tegumento espinoso que le ayuda a romper la piel del hospedador, junto con glándulas que segregan colagenasa y hialuronidasa.
• Esquistosómula: fase que se da en el torrente sanguíneo del hospedador definitivo, se asemeja a una cercaria sin cola (estas la pierden en el proceso de penetración. Se alimentan de plasma
• Adulto: tanto macho como hembra presentan en su extremo anterior una ventosa oral (que se continua con el esófago que se divide, previo a la ventosa ventral, en dos ciegos comunicados con una unión cecal, posterior a esta ventosa; y que continua por todo el cuerpo del animal); y una ventosa ventral, con el poro genital en su interior al que desembocan el útero o el vaso deferente, según sea hembra o macho. Las diferencias sexuales se refieren a:
• Machos: foliáceos, sus regiones laterales forman un pliegue, el canal ginecóforo, que alberga a la hembra hasta la puesta de huevos. Presenta, partiendo del poro genital, situado posterior a la ventosa ventral, un vaso deferente al que desembocan los vasos eferentes, procedentes de los testículos. El final del aparato genital lo forma el cirro esponjoso, un saliente externo que lleva el esperma al poro genital de la hembra.
• Hembras: cilíndricas, se ubican en el canal ginecóforo hasta la ovoposición. Su aparato reproductor comienza en el poro genital y se continúa con el útero, que da a un ensanchamiento (ootipo), donde se da la fecundación. Se continúa con una bifurcación que da al viteloducto (donde desembocan las glándulas vitelógenas) y al ovario, productor de óvulos. El tamaño del aparato genital tiene una significancia taxonómica, pudiendo clasificar en distintas especies según éste ocupe un tercio (S. haematobium), la mitad (S. japonicum), o dos tercios (S. mansoni) del volumen del animal.














Ciclo biológico

El ciclo comienza en el hospedador definitivo (ser humano), con la ovoposición en los plexos mesentéricos y vesicales. La eliminación del huevo (que posee ya el miracidio en su interior) se produce por rotura de los tejidos gracias a las sustancias segregadas por el espolón del huevo. Una vez ya en la orina o en el bolo fecal, sale al exterior por las excretas, teniendo que ir a un medio acuático.

El miracidio sale del huevo por rotura lateral de este a entre 25-30ºC, baja presión de oxígeno y luz. El miracidio, ciliado, busca a su hospedador intermediario, un caracol, siendo cada especie de esquistosoma específica de un género de caracol: Biomphalaria (S. mansoni); Bulinus (S. haematobium); Tricula, Oncomelania, Robertsiella (S. japonicum); penetrando por el pie o los tentáculos rompiendo el tegumento gracias a las glándulas de penetración y una papila de penetración. El miracidio posee unas características que le permiten buscarle (geotropismo negativo y fototropismo positivo). En el punto de penetración pierde los cilios y forma un esporoquiste. De ellos surge una segunda generación que se dirige al hepatopáncreas, donde se formaran la tercera generación de esporoquistes de la que saldrán las cercarias por el moco que suelta el caracol.

Las cercarias (de cola bifurcada) salen siguiendo ritmos cronobiológicos para aumentar las posibilidades de encontrar al hospedador definitivo. En el caso de humanos, abandonan los caracoles por la mañana, cuando se acerca el posible hospedador a beber agua (en roedores, por ejemplo, eso sería por la noche). Además, para favorecer el encuentro, dado que si no lo encuentra antes de las 24 horas morirá, cuentan con una serie de comportamientos: Geotropismo negativo, fototropismo y termotropismo positivo y atracción por sombras, turbulencias y sustancias de la piel. Una vez en ella, penetra activamente mediante rotura con un tegumento espinoso y glándulas de penetración. Procesos como la evaporación de gotas en piel donde esté el parasito aceleran el proceso.

En la penetración se rompe y pierde la cola, a la par que se modifica, formando la esquistosómula. Esta se dispone en las venas de los pulmones, creciendo alimentándose del plasma. Una vez maduras (25 a 48 días postinfección), se desplazan al hígado, donde se encontraran macho y hembra, acoplándose la hembra al canal ginecóforo. Esta no se soltará del macho hasta la ovoposición, donde se cierra el ciclo.

Control:
• Control de excretas y letrinas.
• Dificultar el crecimiento del hospedador intermediario (cambios en sistema de riego, desplazamiento por competencia con otra especie,…)
• Evitar bañarse en sitios contaminados o, secarse inmediatamente tras el baño.





Esquistosomosis y dermatitis causadas por Schistosoma

Las acciones que lleva a cabo el parásito en el hospedador definitivo, más en concreto en el caso humano, dependen, más que del propio parásito en sí, de la fase en la que esté, ya que estas cambian a lo largo del ciclo vital que se da dentro del hospedador. Estas son:
• Furcocercarias: lleva una doble acción conjunta traumática y tóxica, dado que perfora la piel de forma mecánica (posee un tegumento perforador) y mediante la secreción de colagenasa y hialuronidasa, que acaba por romper la piel y la formación de máculas y pápulas.
• Esquistosómulas: lleva a cabo una casi indetectable (salvo alta cantidad de parasitemia) acción expoliadora al alimentarse del plasma.
• Adultos: también dependiente del grado de parasitemia, lleva a cabo una acción mecánica de bloqueo en hígado y/o plexos venosos. También puede producir una acción expoliadora al alimentarse de sangre.
• Huevos: es la acción que más síntomas da y la más importante a nivel clínico. Realiza una doble acción mecánica y tóxica: los huevos, con el miracidio en su interior, exuda por el espolón una serie de antígenos que, bien rompen las paredes de los órganos para salir al lumen, bien desencadenan la formación de granulomas con una inmunopatología asociada: gran inflamación de la zona y la encapsulación del huevo en un granuloma.

La esquistosomosis puede cursar de una manera crónica o aguda. La forma crónica la padecen portadores sanos, no desarrollando patología alguna, produciéndose en personas inmunocompetentes de zonas endémicas. El resto de la población padece la forma aguda de la enfermedad, que recibe nombres vulgares distintos según la especie: Fiebre de Katayana (causada por S. japonicum, propia de Japón), Esquistomatosis toxémica (provocada por S. mansoni, propia de Brasil y Puerto Rico). Los síntomas son:
a) Inicio y hasta los dos meses posteriores a la infección: enfermedad febril de 3 a 10 semanas de duración (hasta la primera ovoposición).
b) Tras el inicio de las ovoposiciones, se dan síntomas abdominales: disentería con sangre y moco, vómitos, nauseas.
c) Durante toda la parasitosis: dolores similares a artritis y otros no específicos; granulomas en pared intestinal o vesical y en tejidos peritoneales; esplenomegalia; y hepatomegalia con fibrosis periportal (que lleva a hipertensión portal y dilatación de venas colaterales).

La esquistosomosis tiende a cronificarse. Su periodo de incubación es de 1 a 7 semanas (según la especie), el periodo patente de 5 a 10 semanas, y el patente, de 5 a 25 años. Además, esta enfermedad posee varios periodos de sintomatología específica, correspondientes a distintos periodos del ciclo vital parasitario:
1. Invasión: se da en el momento de la entrada de las cercarias y la migración de la esquistosómula. Se da una dermatitis en el punto de penetración y asma (esquistosómula en las venas pulmonares)
2. Maduración a adultos: se da en el momento en el que los esquistosomas adultos viajan al hígado y se aparean, tras lo cual la hembra se va al lugar de ovoposición (primera puesta). Se genera fiebre y disentería con sangre y moco. 
3. Ovoposición intensa: la hembra pone los huevos. Se forman granulomas (poliposis intestinal) con esplenomegalia y hepatomegalia.
4. Síntomas asociados a la ovoposición: la acción de las distintas fases del parásito, sobre todo del huevo, generan una fibrosis en órganos donde llegan las formas parasitarias, descompensación hepática, hematemesis e hipertensión portal grave que puede llevar al coma.

Los granulomas se producen como una reacción inmune ante la presencia del parásito, concretamente de los huevos de éste. Así, en torno al huevo se disponen, en capas concéntricas, y de interior a exterior: macrófagos, linfocitos, neutrófilos y células plasmáticas. En algunos casos, consiguen calcificarles. En otros, el huevo consigue sobrevivir.

Esquistosomosis por especie

S. mansoni
S. japonicum
S. haematobium
Localizado en el mesenterio del intestino grueso
Forma granulomas sencillos en intestino grueso (poliposis crónica)
Fibrosis portal, cirrosis y daño hepático
Localizado en el mesenterio del intestino delgado
Forma granulomas múltiples en intestino delgado
Retraso crecimiento físico.
Hepatomegalia (hipertensión portal que lleva al coma)
Esplenomegalia
Ascitis (acumulación de líquido en el peritoneo)
Localización ectópica: cerebro (retraso mental, coma); pulmón (vasculopatía obstructiva)
Hematemesis masiva.
Localizado en los plexos vesicales.
Forma lesiones granuloides en pared vesical
Uropatía obstructiva por hiperplasia de uréteres
Dolor crónico vesical, hematuria, proteinuria.
Fallo renal (hidro o pielonefritits)
Propensión a VIH y cáncer vesical
Lesiones ectópicas: genitales (alteración hormonal (ginecomastia) junto con hepática)

Dermatitis causadas por Schistosoma

Causadas por cercarias del género Schistosoma, parásitas de aves o mamíferos (hospedador definitivo) y caracoles de agua dulce o salada (hospedador intermediario) pero no del ser humano.
Generan en el ser humano una acción, primero, mecánica de ruptura al romper la piel, y una acción tóxica o de hipersensibilidad que lleva a una inflamación de la zona de penetración, la formación de una pápula que forma una pústula. El parasitismo no es próspero: el sistema inmune mata a las cercarias. La sintomatología es localizado en la zona de penetración: prurito, eritema, salpullido que pasa a mácula, pápula y pústula a los tres días de la penetración, remitiendo en una semana (propia respuesta del organismo, junto con antihistamínicos).

La mediad de control frente a estas cercarias es el eliminar mediante secado con una toalla de gotas que puedan quedar sobre la piel a la mayor brevedad posible.



Diagnóstico

• Etiológico: observación de huevos en excretas (heces u orina, según corresponda) o en material de biopsias rectales.
• Indirecto: mediante IFI, ELISA, RIA o la Reacción circumoval o la Reacción de Vogel-Minning:
• Reacción circumoval: estudia la presencia de anticuerpos específicos en suero. Para ello, se dispone una solución con huevos de Schistosoma spp junto a una alícuota de suero sospechoso de tener anticuerpos. Se incuba a 37º durante 30 minutos y se observa: si aparecen estructuras digitiformes cerca del huevo (puntos de precipitación del complejo Ag-AB) se considera positivo.
• Reacción de Vogel-Minning: estudia la presencia de anticuerpos específicos en suero. Para ello, se dispone una solución con cercarias de Schistosoma spp junto a una alícuota de suero sospechoso de tener anticuerpos. Se incuba a 37º durante 30 minutos y se observa: si existe una alteración progresiva (destrucción) de la cutícula del tegumento de la cercaria se considera positivo.

