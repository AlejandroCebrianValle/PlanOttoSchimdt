\chapter{Sistema Internacional de Medida}
Una magnitud física es todo aquello que se puede medir. Serán así magnitudes físicas la longitud, superficie, masa, etc. \textbf{Medir} es determinar una magnitud por comparación con otra que se toma como unidad. El resultado de toda medida es un número y una unidad, el número sin la unidad carece de sentido. Las magnitudes pueden ser añadidas o sustraídas directamente sólo si tienen las mismas unidades.
\section{Error}
\subsection{Error sistemático y accidental}
Una medida perfecta es imposible. Los errores pueden ser:
\begin{itemize}[itemsep=0pt,parsep=0pt,topsep=0pt,partopsep=0pt]
    \item \textbf{Error sistemático}: debido al instrumento de medida. Existe una mala construcción del instrumento o empleo incorrecto. Se aprecian cuando se cambia de instrumento o de método de observación.
    \item\textbf{Error accidental}: de causa múltiple, imposibles de controlar. Se reducen repitiendo varias veces la medida y hallando la media aritmética de los valores obtenidos.
\end{itemize}
\subsection{Error absoluto y relativo}
\begin{itemize}[itemsep=0pt,parsep=0pt,topsep=0pt,partopsep=0pt]
    \item \textbf{Error absoluto}: es la diferencia entre la medida hecha y el valor real de la magnitud medida.
    \item\textbf{Error relativa}: la razón entre el error absoluto de una medición y el valor verdadero de la cantidad de magnitud medida, dado en tanto por uno. Si se multiplica por cien se da en cien por cien.
\end{itemize}

Un ejemplo puede ser la medida mediante un sistema de un patrón conocido de 10 unidades, pero midiendo 9.9996 unidades. Los errores son:
\begin{itemize}[itemsep=0pt,parsep=0pt,topsep=0pt,partopsep=0pt]
    \item Error absoluto: $10 - 9.9996 = 0.0004 U$
    \item Error relativo: $0.0004 / 10 = 0.00004$ en tanto por uno; $0.004\%$
\end{itemize}
\section{Sistema de unidades}
De entre todas las magnitudes susceptibles de medida hay tres, denominadas fundamentales por ser irreductibles, mientras todas las demás están relacionadas con aquéllas de una u otra manera. Puede expresarse a través de las mismas de forma más o menos complicada, siendo magnitudes derivadas. Las unidades fundamentales se definen:
\begin{itemize}[itemsep=0pt,parsep=0pt,topsep=0pt,partopsep=0pt]
    \item \textbf{Longitud}: extensión del espacio que ocupan los cuerpos.
    \item\textbf{Masa}: Cantidad de materia que integran los cuerpos.
    \item\textbf{Tiempo}: duración de un fenómeno.
\end{itemize}

Las propiedades son:
\begin{itemize}[itemsep=0pt,parsep=0pt,topsep=0pt,partopsep=0pt]
    \item La relación entre las magnitudes fundamentales y las derivadas viene expresada por su fórmula de dimensiones.
    \begin{itemize}[itemsep=0pt,parsep=0pt,topsep=0pt,partopsep=0pt]
        \item La superficie se define como el producto de dos longitudes, y por tanto, es una magnitud derivada de la longitud:
            \subitem Fórmula dimensional $(S) = \mbox{longitud} \cdot \mbox{longitud} = L^2$
            \subitem Unidades de superficie $\mbox{metro} \cdot \mbox{metro} = m^2$
        \item El volumen se define como el producto de tres longitudes:
            \subitem Fórmula dimensional $(V) = \mbox{longitud} \cdot \mbox{longitud} \cdot \mbox{longitud} = L^3$
            \subitem Unidades de volumen $\mbox{metro} \cdot \mbox{metro} \cdot \mbox{metro} = m^3$
        \item La unidad de velocidad (v) es la unidad de longitud dividida por la unidad de tiempo (t):
            \subitem Fórmula dimensional $(v) = \mbox{longitud} / \mbox{longitud} =$ longitud/tiempo.
            \subitem Unidades de velocidad $\mbox{metro} / \mbox{segundo} = m/s$
        \item La unidad de densidad es la unidad de masa (m) dividida por la unidad de volumen (V):
            \subitem Fórmula dimensional $(D) =  \mbox{masa} / \mbox{volumen} = M/V$
            \subitem Unidades de densidad $\mbox{kilogramo} \cdot \mbox{metro cúbico} = kg/m^3$
    \end{itemize}
    \item Dos medidas con la misma fórmula de dimensiones son de la misma magnitud y, por tanto, se pueden sumar después de un cambio de unidades.
\end{itemize}
\section{Sistema Internacional de Unidades}
Existen dos tipos de Unidades, las básicas y las derivadas.
\subsection{Unidades básicas}
Son las utilizadas en propiedades físicas o cualidades fundamentales:
\begin{table}[H]
    \centering
    \begin{tabular}{ccc}
        \rowcolor{black}\textcolor{white}{\textbf{Propiedad a medir}}&\textcolor{white}{\textbf{Nombre}}&\textcolor{white}{\textbf{Símbolo}}\\
        Longitud&metro&m\\
        \rowcolor{hiperlightgray}Masa&Kilogramo&kg\\
        Tiempo&segundo&s\\
        \rowcolor{hiperlightgray}Corriente eléctrica&amperio&A\\
        Temperatura&Kelvin&k\\
        \rowcolor{hiperlightgray}Cantidad de sustancia&mol&mol\\
        Intensidad luminosa&candela&kat\\
        \hline
    \end{tabular}
    \caption{Unidades básicas del Sistema internacional}
\end{table}
\subsection{Unidades derivadas}
Son aquellas obtenidas mediante combinación de las unidade básicas (mediante multiplicaciones y/o divisiones), dando ecuaciones más o menos complejas, lo que ha llevado a la adopción, en algunos casos, de nombres y símbolos específicos para facilitar su uso. Algunas con utilidad clínica serían el volumen, densidad, concentración, frecuencia, fuerza, etc. Existen también propiedades físicas adimensionales como indice de refracción o absorbancia. Algunas definiciones son:
\begin{itemize}[itemsep=0pt,parsep=0pt,topsep=0pt,partopsep=0pt]
    \item Metro cúbico ($m^3$): unidad de volumen.
    \item Kilogramo por metro cúbico ($kg/m^{-3}$): unidad de densidad.
    \item Hertz ($s^{-1}$): unidad de frecuencia.
    \item Newton ($m\cdot kg\cdot^{-2}$): unidad de fuerza.
    \item Pascal ($N / m^2$): Unidad de energía, trabajo y calor.
    \item Watio ($J / s = m^2\cdot kg\cdot s^{-2}$): unidad de potencia.
    \item Voltio ($W / A$): unidad de potencial eléctrico.
    \item Katal ($mol / s$): cantidad de catalizador.
\end{itemize}
\subsection{Múltiplos y submúltiplos aceptados}
A la hora de utilizar estas unidades en el laboratorio, tanto básicas como derivadas, pueden ser demasiado grandes o pequeñas para expresar algunas magnitudes. Para solucionarlo se recurre al uso de múltiplos y submúltiplos de las unidades mediante uso de prefijos. Los más utilizados son:
\begin{table}[H]
    \centering
    \begin{tabular}{ccc|ccc}
        \rowcolor{black}\textcolor{white}{\textbf{Prefijo Múltiplos}}&\textcolor{white}{\textbf{Símbolo}}&\textcolor{white}{\textbf{Factor}}&\textcolor{white}{\textbf{Prefijo submúltiplos}}&\textcolor{white}{\textbf{Símbolo}}&\textcolor{white}{\textbf{Factor}}\\
        Peta&P&$10^{15}$&Femto&f&$10^{-15}$\\
        \rowcolor{hiperlightgray}Tera&T&$10^{12}$&Pico&p&$10^{-12}$\\
        Giga&F&$10^{9}$&Nano&n&$10^{-9}$\\
        \rowcolor{hiperlightgray}Mega&M&$10^{6}$&Micro&$\mu$&$10^{-6}$\\
        Kilo&K&$10^{3}$&Mili&m&$10^{-3}$\\
        \rowcolor{hiperlightgray}Hecto&H&$10^{2}$&Centi&c&$10^{-2}$\\
        Deca&D&$10^{1}$& Deci&d&$10^{-1}$\\
        \hline
    \end{tabular}
    \caption{Múltiplos y submúltiplos del Sistema internacional}
\end{table}
\subsection{Conversión al sistema internacional}
Antes de implantarse el sistema internacional, los distinto parámetros que se determinaban se expresaban en unidades pertenecientes a otros sistemas, las unidades convencionales. Será por tanto necesario conocer cómo realizar la conversión de estas unidades a las de SI, para poder interpretar los resultados en los distintos laboratorios, independientemente del sistema de unidades que estén empleando. Para ello es necesario conocer la relación entre ambas unidades.
\section{Formas de expresar concentración}
La materia se puede encontrar bajo diversas formas pudiendo clasificarse en:
\begin{multicols}{2}
    \begin{itemize}
        \item Sustancias puras:
        \begin{itemize}[itemsep=0pt,parsep=0pt,topsep=0pt,partopsep=0pt]
            \item Elementos químicos.
            \item Compuestos puros.
        \end{itemize}
        \item Mezclas:
        \begin{itemize}[itemsep=0pt,parsep=0pt,topsep=0pt,partopsep=0pt]
            \item Mezclas homogéneas.
            \item Mezclas heterogéneas.
        \end{itemize}
    \end{itemize}
\end{multicols}
\subsection{Sustancias puras}
Se trata de sustancias que no pueden reducirse a otras más sencillas en condiciones normales, clasificándose en:
\begin{itemize}[itemsep=0pt,parsep=0pt,topsep=0pt,partopsep=0pt]
    \item \textbf{Elementos químicos}: son sustancias puras formadas por una sola clase de átomos. Cada elemento químico se designa con un nombre determinado y se representa con un símbolo químico.
    \item\textbf{Compuestos puros}: Compuestos formados por átomos de elementos distintos en proporción fija y constante. A cada compuesto se le asigna una fórmula. Se puede definir la formula molecular de una sustancia como una fórmula en la que se indican los elementos químicos que la forman (mediante un símbolo) y unos subíndices que hacen referencia a la proporción en que se encuentra cada elemento en el compuesto.
\end{itemize}
\subsubsection{Átomo}
La materia está constituida por partículas denominadas átomos. El átomo es una estructura compleja dentro de la cual se sitúan las partículas subatómicas, destacando:
\begin{itemize}[itemsep=0pt,parsep=0pt,topsep=0pt,partopsep=0pt]
    \item \textbf{Electrón}: partícula elemental de carga eléctrica negativa y poca masa.
    \item\textbf{Protón}: partícula elemental de carga eléctrica positiva, mayor masa que el electrón y formadora del núcleo.
    \item\textbf{Neutrón}: partícula nuclear de carga neutra.
\end{itemize}
Se puede definir así:
\begin{itemize}[itemsep=0pt,parsep=0pt,topsep=0pt,partopsep=0pt]
    \item \textbf{Número atómico} (Z): es el número de protones en el núcleo de un átomo.
    \item\textbf{Número másico} (A): es la suma de protones y neutrones del núcleo de un átomo.
    \item\textbf{Iśotopo}: átomos de un mismo elemento que difiere en el número de neutrones. Luego los isótopos tiene el mismo número atómico pero distinto número másico.
    \item\textbf{Masa atómica}: número de veces que un átomo de dicho elemento pesa más que una unidad de UMA. Como unidad de masa atómica de referencia (UMA), se estableció como la doceava parte de la masa atómico del isótopo 12 del carbono.
    \item\textbf{Átomo-gramo}: se define como la cantidad de gramos de un elemento que iguala a su peso atómico. Se corresponde con el número de Avogadro ($6.023\cdot 10^{23}$) de átomos del mismo.
\end{itemize}
\subsubsection{Molécula}
La parte más pequeña de un compuesto que mantiene sus propiedades químicas. Se pueden encontrar formadas por átomos iguales o por combinaciones de distintos elementos. Se define como \textbf{masa molecular} el número de gramos en igual valor numérico que la cifra que expresa la masa molecular, que se corresponde con el número de Avogadro. Se define como un \textbf{mol} el equivalente al número de átomo-gramo de una sustancia.
\subsection{Mezclas}
Se obtiene por interposición mecánica de varias sustancias. Se clasifican en:
\begin{itemize}[itemsep=0pt,parsep=0pt,topsep=0pt,partopsep=0pt]
    \item \textbf{Mezclas homogéneas}: presentan una composición uniforme. Es lo denominado disolución.
    \item\textbf{Mezclas heterogéneas}: su composición no es uniforme, denominándose soluciones.
\end{itemize}
\subsubsection{Disolución}
Se puede definir como una mezcla de composición uniforme, no pudiendo diferenciar sus componentes a simple vista. Los componentes de una disolución reciben el nombre de soluto y disolvente. El soluto suele ser el componente minoritario y el disolvente el compuesto de mayor cuantía.

Se pueden clasificar las disoluciones atendiendo al estado físico de sus componentes. Así, tanto soluto como disolvente se pueden encontrar en distitno estado: sólido, líquido o gas. En general, existe un límite para la cantidad de soluto que puede disolverse en un disolvente a determinada temperatura. Cuando este límite se alcanza, se dice que el disolvente está saturado y se tiene una disolución saturada. En las disoluciones en las que el soluto es un sólido y el disolvente un líquido hay que tener en cuenta que la solubilidad viene afectada por:
\begin{itemize}[itemsep=0pt,parsep=0pt,topsep=0pt,partopsep=0pt]
    \item \textbf{Temperatura}: los sólidos suelen ser más solubles en caliente que en frio.
    \item\textbf{Superficie de contacto}: a mayor superficie de contacto sólido-líquido, se disolverá antes el soluto.
    \item\textbf{Agitación}: contribuye a la fragmentación del sólido aumentando de este modo la superficie de contacto.
\end{itemize}