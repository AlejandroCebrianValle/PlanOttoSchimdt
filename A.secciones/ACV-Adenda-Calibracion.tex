\chapter{Técnicas de Calibración}
\section{Consideraciones generales}
Los reactivos alcalinos no se guardan en recipientes de vidrio ya que pueden disolver ciertos tipos de vidrio. Del mismo modo, los disolventes orgánicos pueden atacar determinados plásticos, por lo que se usará material de vidrio. Tanto las micropipetas como los dispensadores automáticos son sistemas volumétricos de alta precisión que se descalibran fácilmente ante la presencia de partículas extrañas que pueden llegar a obturarlos.
\section{Material de vidrio}
Para medir volúmenes en el laboratorio se usan recipientes de vidrio, debido a su gran estabilidad química. El material de vidrio utilizado puede presentar diferentes propiedades como resistencia a calor, fuerza, etc. Se recomienda el uso de vidrio borosilicato. No obstante, no se debe exponer a cambios bruscos de temperatura o presión. Ya sean matraces volumétricos, pipetas, buretas, probetas o dispensadores; todos estos utensilios están diseñados para que pequeños incrementos de volumen den lugar a variaciones grandes en el nivel del líquido, estando eso sí a una temperatura y condiciones estándar. Se pueden encontrar instrumentos volumétricos con distintos tipos de calibración:
\begin{itemize}[itemsep=0pt,parsep=0pt,topsep=0pt,partopsep=0pt]
    \item \textbf{Instrumentos para verter}: pipetas o buretas, permite la adición o extracción de un volumen exacto, estando ya pensado el volumen adherido por humectación.
    \item\textbf{Instrumentos para contener}: como matraces aforado, la cantidad de líquido se encuentra reducida en el volumen no adherido a las paredes del instrumento.
\end{itemize}

En la utilización del material volumétrico, hay que tener en cuenta el error de paralelaje, que consiste en una lectura errónea debido a un defecto de posición del operario. Se debe tener en cuenta que la superficie de un líquido en un volumen estrecho presenta una curvatura o menisco, que se utiliza, su fondo, en el punto de referencia en la calibración y uso del material.
\subsection{Probetas}
Son recipientes de forma cilíndrica provistas de una base que les da estabilidad y un reborde que facilita su vaciado. Van graduadas verticalmente en milímetros y se usan para medidas de poca precisión.
\subsection{Pipetas}
Utensilios de laboratorio que se emplean para transferir volumenes exactos. En la pared de las mismas aparece la capacidad y temperatura a la que pueden ser usadas.
\begin{itemize}[itemsep=0pt,parsep=0pt,topsep=0pt,partopsep=0pt]
    \item \textbf{Pipetas volumétricas}: diseñadas para medir un solo volumen. Cilíndricas, con un ensanchamiento en su parte central. La parte superior lleva gravados datos propios y una señal de calibrado. La parte inferior tiene su extremo estirado para proporcionar un orificio de salida estrecho.
    \item\textbf{Pipetas graduadas}: cilindricas con un extremo inferior, van graduadas para medir cualquier volumen hasta su capacidad máxima. Cuentan con adaptadores en la parte superior para incorporar aspiradores como peras de caucho o adaptadores para la formación de vacio. No deben soplarse para eliminar el líquido adherido por capilaridad, ya que están calibradas teniendo en cuenta ese volumen.
    \item\textbf{Pipetas especiales} o \textbf{micropipetas}: usadas para transferir volúmenes pequeños, de microlitros. Aspecto de jeringuilla y se llenan por succión: tienen un pistón que crea un vacío en una punta de plástico desechable que impide posibles contaminaciones. Así mismo, se tienen capilares de vidrio para sustancias que reaccionan al plástico.
\end{itemize}
\subsection{Buretas}
Útiles destinados a la transferencia de volúmenes exactos de líquidos. Consisten en un tubo de vidrio de diámetro interior uniforme, graduado en la parte exterior y dotado de un dispositivo de cierre en su parte inferior que permite dispensar volúmenes arbitrarios del líquido contenido. Se utilizan, sobre todo, en valoraciones. Existen de volumenes de 2 y hasta de 50 mL, con precisión de 1 a 100 $\mu$L. Las buretas de vidrio disponen, por su parte inferior, de una llave de vidrio esmerilado o material de plástico. Debe girar fácilmente y no tener fugas.

La bureta debe estar limpia y seca. Debe humedecerse un par de veces en la solución empleada. Se llenará hasta un nivel por encima de cero y se enrasa con el volumen extra, abriendo la llave. Comprobar que no hay burbujas en el extremo inferior. Conviene esperar un minuto y comprobar el enrase, para que se escurra la capa de drenado. Se debe manejar teniendo en cuenta el error de paralelaje.
\subsection{Matraces aforados}
Presentan una forma característica en forma de pera y fondo plano. En el cuello del matraz aparece la linea de aforo indicando la capacidad del recipiente. El extremo del matraz se cierra usando para ello un tapón de vidrio esmerilado. Generalmente están calibrados para contener líquidos aunque algunos están diseñados para verter. Son necesarios para preparar soluciones de forma exacta.

Los matraces deben estar limpio, aunque no necesariamente tiene que estar secos. Para preparar una disolución de un producto sólido se empieza pesando la cantidad necesaria de producto, que se traslada a un vaso de precipitados donde se disuelve en una pequeña cantidad del disolvente, traspasándose la matraz aforado y lavando varias veces el vaso de precipitado, vertiendo cada lavado en el matraz hasta enrasar en la marca con cuidado. Se homogeiniza por rotación.

En el caso de líquidos, mediante una pipeta o probeta, se añade el soluto en el fondo del matraz y se añade el disolvente por las paredes, homogeneizandose por rotación.
\subsection{Limpieza y mantenimiento}
Se puede llevar a cabo mediante la limpieza con un cepillo y agua con un detergente. Se enjuaga con agua al menos tres veces con más agua y finalmente con agua destilada. Por lo general, cuanto antes se limpie tras la utilización, mejor suele ser la limpieza.

Otra forma de limpiarlo es mediante la inmersión del material en una solución de limpieza durante 20 a 30 minutos. Se puede aumentar la efectividad mediante mayor temperatura o más tiempo. Se deben lavar con agua destilada después. Así mismo, se puede realizar la limpieza con distinta maquinaria, pero esta se reservará para material de vidrio pero nunca de plástico.
\subsubsection{Limpieza analítica de trazas}
Dependiendo del material, se utilizaran distintos compuestos químicos:
\begin{itemize}[itemsep=0pt,parsep=0pt,topsep=0pt,partopsep=0pt]
    \item \textbf{Trazas metálicas}: se sumergirá el material en ácido nítrico 1 N durante una hora, enjuagando posteriormente con agua destilada.
    \item\textbf{Trazas orgánicas}: se limpia con hipoclorito sódico 1 N, o mediante alcohol, para posteriormente introducirlos en una solución de clorhídrico 1N y enjuagarlo con agua destilada.
    \item\textbf{Trazas de grasa y material hidrófobo orgánico}: se realiza mediante la inmersión del material en solución crómica (10 mg de \ch{Cr_2O_7^2-} en 10 mL de agua y adición hasta 100 mL de \ch{H_2SO_4}) y posterior enjuagado con agua destilada.
\end{itemize}
\subsubsection{Control de calidad}
Diariamente, se debe examinar que ningún material tenga manchas de agua, teniendo que realizar de nuevo la limpieza si existe de todo el material limpiado ese día.

Semanalmente, se debe examinar que no existan ni material húmedo ni que al llenar el material con agua, se produzcan turbulencias extrañas ni que se quede retenido agua al vaciarlo o llenarlo. Una vez a la semana, se añadirá un pequeño volumen de bromosulfotaleína sódica (20 mg/L) a material al azar, para comprobar la presencia de detergente (un color rosado se considera positivo). Se retirará material dañado o astillado.
\section{Material de porcelana}
Se usa cuando se requiere material que soporte elevadas temperaturas (desecación en horno, cenizas, etc.). No se utiliza en laboratorios clínicos. Se deben tener las mismas precauciones que con el material de vidrio.
\section{Material de plástico}
\subsection{Plástico termoresistente}
Se trata de material desechable y uso tiene esencial importancia en laboratorios donde las contaminaciones son críticas. Los materiales son: puntas para micropipetas, tubos de ensayo y centrífuga, tubos eppendorf o microtubos, vasos de precipitados, probetas y otros.

El material se puede esterilizar antes de su uso utilizando autoclave de vapor. Tras el uso se desecha o se reestieriliza, dependiendo si el material ha estado en contacto con material peligroso. El material volumétrico reutilizable se limpia siguiendo las consideraciones del vidrio.
\subsection{Plástico termosensible}
La mayor parte del material de plástico no es termosensible. Este tipo de material, si se vende de forma esteril, se hace mediante radiación gamma.
\begin{itemize}[itemsep=0pt,parsep=0pt,topsep=0pt,partopsep=0pt]
    \item \textbf{Placas petri}: similar a una caja redonda compuesta de dos piezas, se suele utilizar para cultivos. Existen de vidrio o de plástico, siendo estas las más utilizadas.
    \item\textbf{Pipetas pasteur, tubos de ensayo y centrífuga}: tienen las mismas funciones que el material de vidrio, pueden venderse higienicas o estériles.
    \item\textbf{Material volumétrico}: se limpia igual que el material de vidrio.
\end{itemize}
\section{Agua de laboratorio}
Al hacer determinaciones en el laboratorio de análisis se pueden introducir errores por el uso de agua con impurezas, ya sean orgánicas o impurezas. Para eliminarlas, el agua se somete a diversos tratamientos según el grado de pureza requerido.
\begin{itemize}[itemsep=0pt,parsep=0pt,topsep=0pt,partopsep=0pt]
    \item \textbf{Destilación}: el agua es calentada hasta la evaporación, tras lo cual se procede a la condensación. A partir de agua destilada se puede obtener mediante doble destilación mediante un agente oxidante en la segunda que elimine la materia orgánica.
    \item\textbf{Desionización}: se elimina las partículas cargadas presentes en el agua, haciendo pasar ésta a través de una columna con resina de intercambio iónico. Su inconveniente es que no elimina microorganismos.
    \item\textbf{Adsorción}: usan adsorbentes como el carbón, silicatos, etc.
    \item\textbf{Filtración}: la calidad del agua obtenida depende del tamaño del poro del filtro empleado.
\end{itemize}
\section{Instrumentos de pesada}
Los conceptos de masa y peso van unidos, llegando a confundirse por la proporcionalidad existente entre ambas: todos los objetos tienen masa (que es invariable), y en consecuencia, por la acción de la gravedad, peso (que depende de la atracción hacia el centro del cuerpo gravitatorio).  La masa se mide con la balanza para evitar variaciones debidas a modificaciones locales de la aceleración gravitatoria. Según su sensibilidad, se pueden clasificar:
\begin{itemize}[itemsep=0pt,parsep=0pt,topsep=0pt,partopsep=0pt]
    \item \textbf{de precisión}: su sensibilidad se comprende entre los 0.1 y 0.001 gramos.
    \item\textbf{analíticas}: poseen una sensibilidad mayor a 0.0001 gramos.
\end{itemize}
\subsection{Balanzas electrónicas}
Son balanzas monoplato de carga superior. Utilizan una fuente electromagnética para contrabalancear la carga colocada en el platillo.

El platillo se encuentra sobre un cilindro metálico hueco rodeado por una bobina que se encuenta unida al polo de un imán. Una corriente pasa a través de la bobina y produce un campo electromagnético, con el brazo vacío la corriente se ajusta de manera que el nivel del brazo indicador está en posición cero. Cuando se coloca una carga en el plato, un registrador fotoeléctrico acoplado al brazo del soporte cambia de posición, y transmite una corriente a un amplificador que aumenta el flujo de corriente a través de la bobina y restablece el plato a su posición original. Esta corriente es proporcional al peso sobre el plato y produce un voltaje medible que es transformado en un valor numérico por un procesador. De esta manera, la pesada consiste unicamente en depositar objetos y ver el display.

Si para pesar una sustancia se ha de hacer uso de un envase que la contenga, se puede tarar este y añadiendo a posteriori la sustancia obteniendo así el peso neto.
\subsection{Balanza de Mohr-Wesphal}
Se utiliza para obtener la densidad de líquidos. Está formada por un pequeño cilindro de vidrio que se equilibra mediante un contrapeso.

Al introducir el inmersor en distintos líquidos se rompe el equilibrio, recuperándose meidante caballetes de diferentes pesos que se desplazan a lo largo del brazo, en el que se encuentra nueve ranuras espaciadas regularmente. La posición y los pesos de los caballetes, sobre las distintas ranuras, los cuales devuelven la balanza a la primitiva posición de equilibrio para un determinado líquido, indicando la densidad del mismo.
\subsection{Calibración, verificación, exactitud y sensibilidad}
Cuando se instala una balanza es necesario calibrarla. También es necesario su calibración periódicamente (cada dos o tres meses). La calibración de una balanza electrónica implica la pesada de un peso patrón de masa conocido, el valor de la pesada debe corresponder con el valor del patrón. Para calibrar una balanza se buscará exactitud y sensibilidad.

Se dice que una balanza es exacta cuando al añadir masas iguales en un plato, la posición y los valores del display son idéntidos.

Una balanza será sensible si es capaz de distinguir la adición de un pequeño peso  indicándolo mediante cambio en el display o posición del plato. Una balanza tiene una presición de 1 mg si es capaz de diferenciar dos masas que difieren en ello. Una balanza es precsa si al repetir varias veces una misma pesada se obtiene el mismo valor. Es lo que más se busca en una balanza.
\section{Equipos de temperatura}
\subsection{Baños}
Son recipientes con agua cuya temperatura puede ajustarse. Algunos disponen de sistema de agitación y de esta manera permite garantizar que en todo el recipiente la temperatura sea homogénea.
\paragraph{Usos}: incubación de reactivos, muestras y cultivos en medio líquido.
\paragraph{Mantenimiento}: cambiar periódicamente el agua y limpiar la cuba para evitar depósito de sales (usar previamente agua destilada). Comprobar el funcionamiento con un termómetro distinto al instalado en el sistema.
\subsection{Estufas}
Recipiente termostatizado, con cierre hermético y aislamiento del exterior. La temperatura del aire del interior puede ajustarse y mantenerse. Loas hay de diversos tipos según el rango de temperaturas que puedan alcanzar.
\paragraph{Descripción y tipos}: se diferencian tres tipos:
\begin{itemize}[itemsep=0pt,parsep=0pt,topsep=0pt,partopsep=0pt]
    \item \textbf{Estufas de cultivo}: temperaturas que van hasta los 60 C, usándose normalmente a unos 37 C.
    \item\textbf{Estufa}: trabajan a temperaturas entre 50 y 300 C.
    \item\textbf{Horno de mufla}: tienen un rango de temperaturas superior a 1000 C.
\end{itemize}
Cualesquiera fuese el tipo, las partes son:
\begin{itemize}[itemsep=0pt,parsep=0pt,topsep=0pt,partopsep=0pt]
    \item \textbf{Caja interior}: pared metálica con doble aislamiento y compartimentos internos.
    \item\textbf{Puerta frontal}
    \item\textbf{Fuente de calor}
    \item\textbf{Circulador de aire}: por debajo de la columna de circulación. Existen dos tipos de ventilación, la natural (basada en las diferencias de densidad), y la forzada (el aire se mueve por acción de un ventilador situado en el exterior).
\end{itemize} 
\paragraph{Utilidad}: las estufas de baja temperatura (bacteriológicas), hasta los 60 C, son utilizadas en laboratorio clínico e inmunológico. Las de mediana temperatura, tienen utilidad sobre todo para la desdecación de material de vidio lavado en el laboratorio. Cuando se neceiste una estufa de gran volumen, se ha de utilizar una de aire forzado que permitirá mantener una temperatura homogénea en toda la cámara.
\paragraph{Mantenimiento}: Limpiar paredes interiores y comprobación del correcto funcionamiento mediante un termómetro distinto al del sistema. No utilizar para el secado de productos que puedan desprender vapores susceptibles de hacer mezclas peligrosas. No trabajar a una temperatura infoerior a la ambiente. Todas las regulaciones en el interior de la estufa se hacen en sentido ascendente, de menor a mayor temperatura.
\subsection{Neveras y congeladores}
Constan de:
\begin{itemize}[itemsep=0pt,parsep=0pt,topsep=0pt,partopsep=0pt]
    \item \textbf{Cámara interna}:en su interior se enfria el material y permite su distribución.
    \item\textbf{Puertas}: adaptadas de manera que se impida la pérdida de temperatura.
    \item\textbf{Termostato}: selecciona la temperatura deseada, controlando la acción del compresor por medio de una válvula de expansión que regula la entra refrigerante al evaporador.
    \item\textbf{Sistema frigorífico}:
    \begin{itemize}[itemsep=0pt,parsep=0pt,topsep=0pt,partopsep=0pt]
        \item \textbf{Evaporador}:estática (la temperatura interior actúa por gradientes), o de tiro forzado (con el que se mantiene una uniformidad en la temperatura interna de la cámara).
        \item\textbf{Compresor}
        \item\textbf{Condensador}
    \end{itemize}
\end{itemize}
\paragraph{Usos}: permite reducir los procesos químicos a la actividad enzimática, reducir el metabolismo y multiplicación de flora no deseada; aumento de la concentración de solutos en agua residual no cristalizada. En el laboratorio el uso de neveras y congeladores se utilizan a temperaturas de 4 a -20 C, para permitir la conservación de reactivos.
\paragraph{Mantenimiento}: no colocar refrigeradores cerca de zonas de calor ni obstruir las vías de refrigeración. Limpiar los apartaros mediante presión o métodos mecánicos con el aparato apagado. Abrir lo menor posible el aparato. Medir la temperatura de forma diaria y registrarla. Descongelar el aparato según instrucciones del fabricante. 
\subsection{Termómetros}
Los equipos como termómetros de columna líquida, termopares o termómetros de resistencia de platino deben tener una calidad adecuada para cumplir las especificaciones del ensayo. La calibración y trazabilidad de la medida deben estar garantizadas.

Cuando la precisión de la media no tenga efecto directo en el resultado del ensayo, se seguirá manteniendo las condiciones de trazabilidad y acreditación. El laboratorio debe realizar verificaciones independientes de los indicadores de temperatura.
\subsection{Autoclaves}
Los autoclaves deben cumplir las tolerancias de temperatura, especificadas. No se recomienta la utilización de autoclaves provistos sólo de manómetro para esterilizar o descontaminar medios. La estabilidad y la uniformidad de la temperatura, así como el tiempo necesario para alcanzar condiciones de equilibrio en los autoclaves debe verificarse inicialmente y documentarse. La constancia de las características registradas durante la verificación inicial del equipo debe comprobarse y registrarse después de efectuar una modificación o reparación.

El ciclo de esterilización debe tener en cuenta el perfil de calentamiento de la carga. Deben darse instrucciones claras de funcionamiento basadas en los perfiles de calentamiento para cada uso. El control de la temperatura puede realizarse mediante el uso de un termopar y un aparato de registro, un termómetro de temperatura máxima o por observación directa y registro de temperatura máxima. Se ha de comprobar así mismo, mediante uso de indicadores, la efectividad del ciclo.


