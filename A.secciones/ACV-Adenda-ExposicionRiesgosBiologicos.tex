\chapter{Riesgos de trabajo en Laboratorio}
\section{Riesgos específicos a agentes biológicos}
El personal de los laboratorios está expuesto a ciertos riesgos biológicos, que son bacterias, virus, hongos, parásitos mono o pluricelulares y animales de experimentación. La protección frente a los riesgos relacionados con la exposición a agentes biológicos está regulado por el Real Decreto 664/1997 (RD 664/97 de aquí en adelante) y la adaptación contenida en la Orden de 25 de marzo de 1998. Este RD 664/97 establece para el trabajo con microorganismos, así como para aquellas actividades que implican la manipulación de animales vertebrados infectados natural o deliberadamente, cuatro niveles de seguridad o contención biológica. Así mismo, las Notas Técnicas de Prevención (NTP) realizadas por el Insituto Nacional de Seguridad e Higiene en el Trabajo contienen informaciones específicas sobre los riesgos que se pueden producir al trabajar en el laboratorio y recomendaciones a seguir. No sólo se deben aplicar medidas específicas sino también deben aplicarse las medidas generales de seguridad relativas a la higiene personal, al trabajo seguro y buenas prácticas frente al riesgo biológico utilizando los equipos de protección adecuados.
\subsection{Principios básicos de seguridad biológica}
\subsubsection{Terminología (RD 664/97)}
\begin{itemize}[itemsep=0pt,parsep=0pt,topsep=0pt,partopsep=0pt]
    \item \textbf{Agentes biológicos}: microorganismo, con inclusión de organismos modificados genéticamente, cultivos celulares y endoparásitos humanos susceptibles de originar cualquier tipo de infección, alergia o toxicidad.
    \item\textbf{Microorganismo}: toda entidad microbiológica, celular o no, capaz de reproducirse o de transferir material genético.
    \item\textbf{Cultivo celular}: resultado del crecimiento \textit{in vitro} de células obtenidas de organismos pluricelulares.
    \item\textbf{Peligro}: todo aquello que puede producir un daño o un deterioro de la calidad de vida individual o colectiva de las personas.
    \item\textbf{Daño}: consecuencia producida por un peligro sobre la calidad de vida individual o colectiva de las personas.
    \item\textbf{Riesgo}: probabilidad de que ante un determinado peligro se produzca un cierto daño, pudiendo por ello cuantificarse.
    \item\textbf{Desinfección} (OMS): eliminación de agentes infecciosos que están fuera del organismo por medio de la exposición directa a agentes químicos o físicos.
    \item\textbf{Contaminación} (OMS): presencia de un agente infeccioso en la superficie del organismo; también en la vestimenta, ropa de cama, juguetes, instrumental quirúrgico, apósitos u otros objetos inanimados o sustancias, incluyendo el agua y alimentos.
    \item\textbf{Esterilización} (OMS): destrucción de toda forma vida por calor, radiación, gas o tratamiento químico.
    \item\textbf{Limpieza}: eliminación según, o un detergente adecuado, o por el empleo de aspiradora, de agentes infecciosos y sustancias orgánicas de superficies en las cuales estos pueden encontrar condiciones adecuadas para sobrevivir o multiplicarse.
\end{itemize}
\subsubsection{Clasificación de peligrosidad}
\begin{itemize}[itemsep=0pt,parsep=0pt,topsep=0pt,partopsep=0pt]
    \item\textbf{Grupo 1}: microorganismos comunes que rara vez ocasionan enfermedades en individuos inmunocompetentes, no requiere medidas de seguridad.
    \item\textbf{Grupo 2}: microorganismo causante de enfermedades, cuyo manejo implica un riesgo en el laboratorio, aunque es raro que puedan causar epidemias. Su prevención o tratamiento es sencillo y efectivo. No requieren medidas especiales de seguridad, pero sí cabinas de bioseguridad si pueden producir aerosoles.
    \item\textbf{Grupo 3}: microorganismos causantes de infecciones graves, que suponen un riesgo serio para el trabajador del laboratorio. Existe el peligro de diseminación de la infección. Su profilaxis y tratamiento son efectivos. Debe trabajarse en cabinas de bioseguridad, sobre todo si se trata de cultivo puro.
    \item\textbf{Grupo 4}: microorganismos causantes de infecciones graves que suponen también un riesgo serio para el trabajador del laboratorio. En estos casos es aún mayor el peligro de diseminación comunitaria. No existe profilaxis ni tratamiento efectivos. Debe trabajarse con cabinas de seguridad.
\end{itemize}

El aire debe fluir de las zonas limpias a las sucias (áreas de trabajo) y de ahí al exterior, o volver a circular tras ser filtrado. En laboratorios que trabajen con microorganismos de los grupos III y IV, el aire debe ser filtrado con filtros HEPA.
\subsubsection{Niveles de contención}
La seguridad biológica se fundamenta en técnicas de laboratorio, equipos de seguridad o barreras primarias y diseño de instalaciones o barreras secundarias:
\begin{itemize}[itemsep=0pt,parsep=0pt,topsep=0pt,partopsep=0pt]
    \item \textbf{Técnicas de laboratorio}: deben seguir prácticas y técnicas estándar microbiológicas para contener riesgos (Manual de seguridad biológica).
    \item\textbf{Equipos de seguridad}: dispositivos  que garanticen la seguridad (i.e. prendas de protección personal.)
    \item\textbf{Diseño y construcción de la instalación}: barreras que dependen del tipo de agente infeccioso que se manipule. Dentro de ellas se incluyen la separación de las zonas a las que tiene acceso el público, disponibilidad de sistemas de descontaminación, filtrado de aire de salida al exterior, etc.
\end{itemize}

El término contención hace referencia a los métodos que se emplean para hacer seguro el manejo de materiales infecciosos en el laboratorio. La contención tiene como finalidad reducir al mínimo la exposición del personal de los laboratorios , otras personas y el entorno a agentes potencialmente peligrosos. De todos, el mayor es el nivel 4, para procesar aquellos agentes patógenos o infectocontagiosos que producen alta mortalidad y para el que no existe tratamiento y/o es poco fiable.

\begin{tabular}{N{3.5cm}N{3.5cm}N{3.5cm}N{3.5cm}}
    \rowcolor{black}\textcolor{white}{\textbf{Grupo}}&\textcolor{white}{\textbf{Riesgo individual}}&\textcolor{white}{\textbf{Riesgo comunitario}}&\textcolor{white}{\textbf{Riesgo comunitario}}\\
    I&Bajo&Bajo &Nivel 1\\
    \rowcolor{hiperlightgray}II&Moderado&Limitado&Nivel 2\\
    III&Alto&Bajo o Limitado&Nivel 3\\
    \rowcolor{hiperlightgray}IV&Alto&Alto&Nivel 4\\
    \hline
\end{tabular}
\subsection{Riesgos específicos de exposición a bacterias}
En el RD 664/1997 se detalla una lista de especies, de las que se pueden extraer los más comunes ante su manipulación. Independientemente del agente bacteriano a manipular, debe ser una práctica universal la utilización de bata y guantes, practicas de higiene personal correctas y lavado frecuente de manos.
\subsubsection{\textit{Bacillus anthracis}}
Bacilo grande, Gram positivo, inmóvil, aerobio y esporulante (espora como forma de resistencia a desecación y altas temperaturas). Las técnicas de desinfección eliminan la forma vegetativa pero no la de resistencia. La virulencia de este microbio va ligada a la cápsula (las cepas no capsuladas no son virulentas), reforzándose con la producción de toxinas que complementan los mecanismos antifagocitarios. El reservorio está constituido por herbívoros que lo eliminan en heces y orina. La enfermedad que produce es el carbunco, que afecta al ganado y al ser humano de tres maneras:
\begin{itemize}[itemsep=0pt,parsep=0pt,topsep=0pt,partopsep=0pt]
    \item \textbf{Cutánea}: más frecuente, provoca la formación de una pústula maligna acompañada de fiebre.
    \item\textbf{Pulmonar} e \textbf{intestinal} : producen una septicemia, siendo el caso más grave.
\end{itemize}

Trabajo en el laboratorio:
\begin{itemize}[itemsep=0pt,parsep=0pt,topsep=0pt,partopsep=0pt]
    \item\textbf{Peligros en el laboratorio}: puede encontrarse en la sangre y otros fluidos biológicos, en exudados de lesiones cutáneas, orina, heces y diferentes productos producidos de animales infectados. El principal peligro para el personal se produce por contacto directo e indirecto de la piel con cultivos y superficies contaminadas, inoculación parenteral y exposición a aerosoles infecciosos.
    \item\textbf{Precauciones}: En general, para las diferentes actividades se recomienda un nivel de contención 3. El laboratorio de experimentación animal, con animales potencialmente infectados, también deben cumplir con las medidas asignadas al nivel de seguridad 3. Además, los manuales de seguridad recomiendan medias de seguridad adicionales para trabajar con este agente por su posible uso en terrorismo biológico.
\end{itemize}
\subsubsection{\textit{Clostridium tetani}}
Bacilo Gram positivo, anaerobio estricto, no capsulado, móvil y productor de esporas (endosporas) resistentes al calor, humedad, luz y antisépticos. Sintetiza una potente neurotoxina responsable de la enfermedad del tétanos, siendo esta muy termolabil y se inactiva en presencia de oxígeno. Su hábitat es el suelo, especialmente tierra de cultivo y el intestino humano o animal.
\begin{itemize}[itemsep=0pt,parsep=0pt,topsep=0pt,partopsep=0pt]
    \item\textbf{Peligros en el laboratorio}: los principales peligros son la inoculación parenteral y la ingesta de la toxina. Se desconoce si la toxina se puede absorber a través de mucosas y la exposición a aerosoles. 
    \item\textbf{Precauciones}: se recomienda un nivel de contención 2 para la manipulación de cultivos y toxina. Para el personal de laboratorio el riesgo de contraer la enfermedad es bajo y existe vacuna y suero con anticuerpos frente a la toxina.
\end{itemize}
\subsubsection{\textit{Escherichia coli}}
Bacilo Gram negativo, enterobacteria, no esporulado, aerobio facultativo, oxidasa negativo con requerimientos nutritivos sencillos y relativamente resistente a agentes externos. Es un microorganismo de elección para investigación genética y biotecnológica. Se reconocen dos tipos de cepas de \textit{E. coli}: cepas capaces de adherirse a la superficie mucosa del intestino delgado y cepas no patógenas incapaces de adherirse al intestino delgado o producir enterotoxinas.
\begin{itemize}[itemsep=0pt,parsep=0pt,topsep=0pt,partopsep=0pt]
    \item\textbf{Peligros en el laboratorio}: varian en función de la cepa que se manipula. Las fuentes de contaminación son alimentos contaminados y heces. Se pueden distinguir cuatro grupos principales de cepas capaces de producir enterotoxinas:\textit{E. coli} enterohemorrágicas (ECEH) o \textit{E. coli} verocitotóxicas (ECVT):
    \begin{itemize}[itemsep=0pt,parsep=0pt,topsep=0pt,partopsep=0pt]
        \item \textbf{\textit{E. coli} enteroinvasiva} (ECEI): las fuentes de contaminación son alimentos contaminados, heces, agua, materiales contaminados.
        \item \textbf{\textit{E. coli} enteropatógenas} (ECEP): las fuentes de contaminación son heces.
        \item \textbf{\textit{E. coli} enterotóxica} (ECET): las fuentes de contaminación son alimentos contaminados, heces, agua, materiales contaminados.
    \end{itemize}
    \item\textbf{Precauciones}: Se recomienda un nivel de contención 2 para la manipulación de cultivos o de material contaminado. Se recomienda para cepas verocitotóxicas (ej. 0157:H7) un nivel de contención 3. Las medidas higiene personal deben ser estrictas, dado que la via de entrada es oral.
\end{itemize}
\subsubsection{\textit{Francisella tularensis}}
Bacilo Gram negativo, aerobio, con requerimientos nutritivos específicos, necesita agar sangre con cisteína para su crecimiento. En el RD 664/97 figuran dos tipos: \textit{F. tularensis} A (aislada en roedores y artrópodos, muy virulenta en ratón y humano) y \textit{F. tularensis} B (aislado en agua y animales marinos, poco virulenta para ratón y ser humano).
\begin{itemize}[itemsep=0pt,parsep=0pt,topsep=0pt,partopsep=0pt]
    \item\textbf{Peligros en el laboratorio}: el principal peligro lo constituye el personal manipulando material de riesgo biológico contaminado con este microorganismo. La infección se produce por el contacto directo de piel o mucosas con material infectado, inoculación parenteral accidental, ingestión, exposición a aerosoles o gotículas infecciosas.
    \item\textbf{Precauciones}: Se debe manipular en laboratorios con un nivel de contención 3 (en el caso del tipo A), o de tipo 2, para el tipo B. Los manuales de bioseguridad de USA recomiendan medidas adicionales por su potencial uso con fines terroristas.
\end{itemize}
\subsubsection{\textit{Micobacterium spp}}
Género de bacilos Gram positivo, rectos o ligeramente curvados, aerobios y no esporulados. No se tiñen con la tinción de Gram pero sí con la tinción de Zielh-Neelsen (ácido-alcohol resistente), por el alto contenido en lípidos en su superficie celular. Este género comprende bacterias que producen varias  enfermedades: lepra (\textit{M. leprae}), tuberculosis humana (\textit{M. tuberculosis}, \textit{M. bovis}, \textit{M. africanum}) y otras enfermedades con varias denominaciones, que se clasifican en:
\begin{itemize}[itemsep=0pt,parsep=0pt,topsep=0pt,partopsep=0pt]
    \item \textbf{Enfermedades pulmonares paratuberculosas}: \textit{M. kansaii}, \textit{ M. avium}.
    \item \textbf{Linfadenitis}: \textit{M. scrofulaceum}, \textit{M. fortuitum}, \textit{M. kansaii}.
    \item \textbf{Úlceras cutáneas}: \textit{M. ulcerans}, \textit{M. marinum}, \textit{M. fortium}, \textit{M. chelonei}.
\end{itemize}
\begin{itemize}[itemsep=0pt,parsep=0pt,topsep=0pt,partopsep=0pt]
    \item \textbf{Peligros en el laboratorio}: manipulación de muestras ambientales (suelos o aguas) o tejidos, exudados o esputos de enfermos. Debe extremarse las precauciones en técnicas que generen aerosoles para evitar la inhalación de gérmenes asociados a enfermedades pulmonares.
    \item \textbf{Precauciones}: Nivel de contención 2 para todos los agentes, con excepción de \textit{M. ulcerans} (nivel de contención 3).
\end{itemize}
\subsubsection{\textit{Micobacterium tuberculosis}  y \textit{M. bovis}}
Bacilos Gram positivo, rectos o ligeramente curvados, aerobios y no esporulados. Se presentan en agrupaciones de dos a tres bacilos, en una conformación que recuerda a caracteres chinos. La vía de entrada de estos microorganismos es por inhalación de aerosoles (se puede transmitir mediante toses sobre personal no infectado), inoculación parenteral, contacto directo con mucosas, ingestión de los agentes patógenos.
\begin{itemize}[itemsep=0pt,parsep=0pt,topsep=0pt,partopsep=0pt]
    \item \textbf{Peligros en el laboratorio}: manipulación de muestra para diagnóstico (esputos, orina, aspirado gástrico o bronquial, líquido cefalorraquideo y pleural)'así como tejidos infectados
    \item \textbf{Precauciones}: Nivel de contención 3 y prácticas higiénicas adecuadas para la manipulación de muestras potencialmente contaminadas. Los laboratorios de experimentación animal deben adoptar también un nivel 3 de contención, con especial énfasis para primates no homínidos y roedores.
\end{itemize}
\subsection{Riesgos específicos de exposición a virus}
El RD 664/1997 y las recomendaciones de la CDC recomiendan el uso de EPIs (guantes, mascarillas, protectores oculares y/o faciales, batas y ropa de trabajo; prevención la exposición a fluidos potencialmentes contaminados) y la elección de una barrera protectora adecuada al procedimiento a utilizar. Especial mención el cuidado con las agujas contaminadas (no deben reencapsularse) y la formación de aerosoles.
\subsubsection{Virus de la Hepatitis A y Hepatitis E}
Estos virus constituyen un riesgo para personal que maneja animales (especialmente primates no homínidos). El virus de la hepatitis E (VHE) supone un riesgo menor que el Virus de la Hepatitis A (VHA), salvo en gestantes por riesgos teratogénicos.
\begin{itemize}[itemsep=0pt,parsep=0pt,topsep=0pt,partopsep=0pt]
    \item \textbf{Peligros en el laboratorio}: el agente infeccioso está en heces, saliva y sangre. La ingestión y su contacto con suspensiones constituyen el principal riesgo, así como la exposición a material contaminado. No se ha encontrado referencias a exposición a aerosoles. 
    \item \textbf{Precauciones}: nivel de contención 2, facilitar EPIs para actividades que impliquen la manipulación de heces y otros materiales; así como para personal de laboratorio que trabaje con animales potencialmente infectados. Se recomienda vacunar al personal expuesto al VHA.
\end{itemize}
\subsubsection{Otros Virus de la Hepatitis: VHB, VHC, VHD}
La hepatitis B es una de las enfermedades infecciosas más comunes entre el personal de laboratorio. La exposición al Virus de la Hepatitis C es mayor en personal sanitario siendo la vía parenteral la más frecuente. El Virus de la Hepatitis D (VHD) necesita la presencia del Virus de la Hepatitis B (VHB), por lo que la protección frente al VHB protege también frente al VHD.
\begin{itemize}[itemsep=0pt,parsep=0pt,topsep=0pt,partopsep=0pt]
    \item \textbf{Peligros en el laboratorio}: el VHB se encuentra en sangre y componentes de la misma y en otros fluidos biológicos, siendo las vías de exposición la inoculación parental y exposición a piel y mucosas a fluidos infectados, pudiendo detectarse durante varios días fuera del organismo. El VHC se detecta en sangre y suero sanguíneo, principalmente.
    \item \textbf{Precauciones}: nivel de contención 2 para las actividades que manipulan tejidos o fluidos corporales potencialmente infectados. En el caso de manejo de animales, facilitar EPIs adecuados. Para el caso de manipulación con elevado riesgo de generación de aerosoles, utilizar un nivel de contención 3. Se recomienda vacunar al personal expuesto al VHB. Es imprescindible el uso de ropa de laboratorio y guantes para el trabajo con el contacto directo con material o individuos potencialmente infecciosos.
\end{itemize}
\subsubsection{Herpesvirus simiae B}
El virus B se manifiesta bajo la forma de infección laente en monos, reactivándose de forma espontánea.
\begin{itemize}[itemsep=0pt,parsep=0pt,topsep=0pt,partopsep=0pt]
    \item \textbf{Peligros en el laboratorio}: la mayora parte de los casos se presentan cuando se trabaja directamente con animales vivos importados de paises de origen. Las infecciones conocidas se produjeron durante la manipulación de cultivos celulares de monos infectados, sobre todo del tejido renal de \textit{Macaca rhesus}. Otros casos se debieron por vía parenteral, por punción o corte con cristales infectados.
    \item \textbf{Precauciones}: nivel de contención 4 para trabajos con materiales o animales infectados. Se debe utilizar un material y protecciones personales para la prevención de arañazos y mordeduras.
\end{itemize} 
\subsubsection{Herpesvirus Simplex (1 y 2), Herpesvirus hominis}
El Virus Herpes Simplex 1 (HSV-1), o virus del herpes labial, es una primoinfección generalmente beningna y que la reactivación genera vesículas en la mucosa bucal y borde mucocutáneo. HSV-2 o virus del herpes genital, supone una enfermedad venérea y la mayor parte de las infecciones en neonatos. El virus puede ser eliminado por las mucosas en periodos de hasta 12 días (HSV-2) o 7 semanas (HSV-1).
\begin{itemize}[itemsep=0pt,parsep=0pt,topsep=0pt,partopsep=0pt]
    \item \textbf{Peligros en el laboratorio}: existe mayor riesgo en el personal que maneja material clínico. El peligro está relacionado con ingestión, inoculación parenteral, exposición a salpicaduras en mucosa ocular, nasal o bucal, e inhalación de aerosoles.
    \item \textbf{Precauciones}: nivel 2 de contención para los trabajos con cultivos con material potencialmente infectado. Se utilizará ropa de laboratorio y guantes si existe riesgo directo con material infeccioso.
\end{itemize}
\subsubsection{Virus de la gripe (A, B, C)}
No se han declarado casos de la enfermedad contraída por personal de laboratorio diferentes de la población general.
\begin{itemize}[itemsep=0pt,parsep=0pt,topsep=0pt,partopsep=0pt]
    \item \textbf{Peligros en el laboratorio}: el virus está presente en secreciones respiratorias de individuos infectados, especialmente en la cloaca de determinadas aves. La vía de entrada principal es la inhalación de aerosoles. Se transmite por contacto directo con gotículas infecciosas y por dispersión aérea. Puede persistir en mucosidad seca.
    \item \textbf{Precauciones}: nivel de contención 2 para la recepción y preparación de muestras para diagnóstico o para material procedente de autopsias. Se recomienda vacunar al personal expuesto, aunque solo existe para los tipos A y B.
\end{itemize}
\subsubsection{Virus de la Coriomeningitis Linfocítica (CML)}
Las infecciones por este virus son poco frecuentes, sobre todo motivadas por el contacto con hámsters domésticos infectados o roedores y monos de laboratorio.
\begin{itemize}[itemsep=0pt,parsep=0pt,topsep=0pt,partopsep=0pt]
    \item \textbf{Peligros en el laboratorio}: el virus está presente en sangre, líquido cefalorraquídeo, orina, secreciones rinofaríngeas, heces y tejidos infectados. La contaminación se produce por inoculación parenteral, contaminación a través de mucosas o lesiones cutáneas, tejido o líquidos procedentes de animales infectados, exposición a aerosoles o por la manipulación de cultivos celulares infectados.
    \item \textbf{Precauciones}: nivel de contención 3 y precauciones especiales del personal para trabajos que comportan manipulación del virus, animales potencialmente infectados y cepas nuerotrópicas. Nivel de contención 2 para el resto de cepas.
\end{itemize}
\subsubsection{Poliovirus}
Son pocos frecuentes las infecciones por este virus y soló representan peligro para el personal no vacunado.
\begin{itemize}[itemsep=0pt,parsep=0pt,topsep=0pt,partopsep=0pt]
    \item \textbf{Peligros en el laboratorio}: este virus se encuentra en heces y secreciones de garganta de individuos infectados (personas, primates no homínidos). La ingestión o inoculación parenteral de fluidos o tejidos infecciosos son las principales vías de infección.
    \item \textbf{Precauciones}:  Nivel de contención 2 (muestras clínicas y laboratorios de experimentación animal). Todo el personal debe estar vacunado y tener evidencia serológica de inmunidad frente a los tres tipos de poliovirus.
\end{itemize}
\subsubsection{\textit{Poxviridae}}
Los virus de este género (virus de la viruela humana, vaca, mono, de la vacuna y tanapox) causan infecciones con lesiones cutáneas vesiculares, pústulas circunscritas con contenido pruriginoso. La viruela humana es la única enfermedad erradicada, todo gracias a la vacuna. La epidemiología demuestra que las infecciones a humanos son debidas a contacto con monos infectados de viruela del mono.
\begin{itemize}[itemsep=0pt,parsep=0pt,topsep=0pt,partopsep=0pt]
    \item \textbf{Peligros en el laboratorio}: el agente se encuentra en fluidos y costras de las lesiones, secreciones respiratorias u otros tejidos infectados. El principal peligro para el personal de laboratorio y animalario son las exposiciones a aerosoles, contacto directo con piel y mucosas, así como la inoculación parenteral.
    \item \textbf{Precauciones}:  nivel de contención 2 para todas las manipulaciones con poxvirus diferentes del virus de la viruela, único riesgo de infección para el personal expuesto. Se exige vacunación del personal, no exposición a individuos inmunodeprimidos y manejo en cabinas de seguridad de clase I o II .
\end{itemize}
\subsubsection{Virus de la rabia (\textit{Rhabdoviridae})}
La enfermedad, poco frecuente en humanos, se transmite por contacto directo con saliva y sangre humanas (mordedura) de perros, mofetas, zorros, gatos, murciélagos, chacales y monos.
\begin{itemize}[itemsep=0pt,parsep=0pt,topsep=0pt,partopsep=0pt]
    \item \textbf{Peligros en el laboratorio}: la principal forma de infección del personal se debe a la manipulación natural o experimental de animales infectados o sus tejidos. Se han descrito escasas infecciones por inhalación de material infectado.
    \item \textbf{Precauciones}:  nivel de contención 2 y recomendaciones especiales para personal que manipule materiales potencialmente infecciosos. Se recomienda nivel de contención 3, equipos de precaución personal y otras recomendaciones para operaciones que manejen tejidos con gran carga infectiva y/o capaces de generar grandes cantidades de aerosoles. En cualquier caso, se recomienda vacunación a personal y animales de laboratorio.
\end{itemize}
\subsubsection{Retrovirus: VIH y VIS}
Los retrovirus se estudian por ser modelos para estudio de bases moleculares del cáncer y de la aparición del SIDA. Se transmite por exposición directa a fluidos biológicos infectados, contacto sexual, vía parenteral y placentaria.
\begin{itemize}[itemsep=0pt,parsep=0pt,topsep=0pt,partopsep=0pt]
    \item \textbf{Peligros en el laboratorio}: los riesgos provienen de la posibilidad de inoculación accidental o de contacto con piel y mucosas con fluidos infectados.
    \item \textbf{Precauciones}:  en caso de muestras no concentradas, se puede realizar en un nivel de contención 2. Cultivos celulares y derivados concentrados, se harán en nivel de contención 3.
\end{itemize}
\subsubsection{Virus de la estomatitis vesicular}
Riesgo para trabajadores en contacto con ganado bovino, estando presente en fluidos vesiculares, tejidos y sangre de animales infectados y sangre y garganta de seres humanos.
\begin{itemize}[itemsep=0pt,parsep=0pt,topsep=0pt,partopsep=0pt]
    \item \textbf{Peligros en el laboratorio}: los principales resigos son la exposición a aerosoles y gotículas, contacto de piel con mucosas y tejidos o fluidos infectados e inoculación parenteral.
    \item \textbf{Precauciones}:  nivel de contención 3 para aislados virulentos. Las cepas de laboratorio (virulencia atenuada) basta con un nivel de contención 3.
\end{itemize}
\subsubsection{Priones: Kuru, Creutzfeld-Jacob, y otros}
Estas partículas infecciosas (proteínas) se asocian a encefalopatísa espongiformes transmisibles: Creutzfeldt-Jacob, Encefalopatía Espongiforme Bovina, Kuru, síndrome de Gerstmann-Sträussler-Scheynker. El agente se encuentra principalmente en tejido nervioso (kuru) o en otros órganos  tales como bazo y sistema linfático, pulmones, riñones, sangre... (otros priones).
\begin{itemize}[itemsep=0pt,parsep=0pt,topsep=0pt,partopsep=0pt]
    \item \textbf{Peligros en el laboratorio}: estas infecciones no suponen riesgo. La manipulación de tejidos contaminados constituye un peligro para laboratorios de anatomía patológica ante inoculaciones accidentales, especialmente en tejido nervioso. No se conoce riesgo por aerosoles o exposición directa a mucosas.
    \item \textbf{Precauciones}:  se recomienda nivel de contención 3 para todos los priones a excepción del prión del \textit{Scrapie}, que es suficiente con un nivel de contención 2.
\end{itemize}
\subsection{Riesgos específicos de exposición a hongos}
Los hongos son organismos eucariotas, heterótrofos, uni o pluricelulares, de naturaleza mayoritariamente saprofita, aunque algunos puedan ser patógenos oportunistas, generando distintos tipos de micosis:
\begin{itemize}[itemsep=0pt,parsep=0pt,topsep=0pt,partopsep=0pt]
    \item \textit{\textbf{Superficiales}}: afectan al cabello y capas superficiales de la epidermis.
    \item \textit{\textbf{Cutáneas}}: afectan a epidermis, cabello y uñas.
    \item \textit{\textbf{Subcutáneas}}: afectan a piel, tejido subcutánea, fascia y huesos.
    \item \textit{\textbf{Generalizadas o profundas}}: afectan a órganos internos.
\end{itemize}
La infección en el laboratorio ocurre por la inhalación o inoculación de una de las formas reproductoras. Estos seres tienen varias opciones de reproducción, ya sean sexual, parasexual  o asexual.

Las levaduras son hongos unicelulares (2 a 4 $\mu$m) que se reproducen fundamentalmente por gemación. Los mohos son hongos pluricelulares que crecen formando hifas (estructuras tubulares) que conforman un micelio, que puede ser vegetativo (pegado al sustrato) o aéreo o reproductor, donde se forman las esporas.

La reproducción asexual es el crecimiento a partir de un micelio pseudomicelio primitivo, sin conjugación nuclear ni reproducción cromática, denominándose a los hongos que se reproducen así <<\textit{hongos imperfectos}>>. Este tipo de reproducción puede ser:
\begin{itemize}[itemsep=0pt,parsep=0pt,topsep=0pt,partopsep=0pt]
    \item \textbf{Gemación}: formación de una yema en un punto de la célula madre, a partir de donde la célula hija aumenta de tamaño, se separa de la progenitora y da lugar a otras células.
    \item \textbf{Esporulación-Germinación}: se forman esporas (formas de resistencia) dentro de la célula madre que germinarán en un medio adecuado. Si se desarrollan directamente en la célula vegetativa, reciben el nombre de talosporas. Si se desarrollan en estructuras especializadas se denominan de diferentes formas: conidias, artrosporas, blastosporas, clamidosporas y esporangiosporas.
    \item \textbf{Fragmentación}: las hifas se fragmentan y cada fragmento da lugar a una nueva colonia, siendo un procedimiento común en subcultivos de laboratorio.
\end{itemize}

La reproducción sexual consiste en un ciclo de formación de esporas por fusión de dos núcleos haploides sexualmente distintos. Aunque no es común en hongos patógenos humanos, existen formas de hongos de reproducción sexual que infectan al ser humano (pudiendo recibir dos nombres). Los principales tipos de esporas sexuales son ascosporas, zigosporas y oosporas.

La reproducción parasexual ocurre en el caso de hifas que se unen sin fusión nuclear posterior, dando lugar a un heterocarión de núcleos haploides, pudiéndose conjugarse núcleos y aparece un diploide heterozigótico (i.e. \textit{Aspergillus nidulans}).
\subsubsection{\textit{Blatomyces dematitidis} (o \textit{Ajellomyces dermatitidis})}
Las infecciones de este hongo se manifiestan como micosis granulomatosa pulmonar (crónica o aguda) o cutánea. La infección pulmonar ocurre tras la inhalación de conidios, generando una neumonía o una blastomicosis pulmonar. Las lesiones cutáneas se producen por inoculación accidental de un tejido o cultivo de \textit{B. dermatitidis} en fase levaduriforme. Los granulomas aparecen en cara y partes distales de miembros. No es transmisible de ser vivo a ser vivo de forma directa.
\begin{itemize}[itemsep=0pt,parsep=0pt,topsep=0pt,partopsep=0pt]
    \item \textbf{Principlaes focos de infección}: la forma de levadura está presente en tejidos infectados y en muestras clínicas, mientras que el micelio se haya en cultivos.
    \item \textbf{Peligros en el laboratorio}: la inoculación parenteral accidental de tejidos infectados o de cultivos con la forma levadura causa granulomas locales, mientras que la exposición a aerosoles con conidios infecciosos da lugar a la forma pulmonar.
    \item \textbf{Precauciones recomendadas}: nivel de contención 3 para manipulación de cultivos, muestras clínicas, tejidos o animales infectados.
\end{itemize}
\subsubsection{Coccidiodes immitis}
Produce micosis generalizadas que empiezan por infecciones respiratorias, con una primoinfección asintomática o de tipo gripal; una afección granulomatosa evolutiva con lesiones pulmonares y abscesos en todo el cuerpo.
\begin{itemize}[itemsep=0pt,parsep=0pt,topsep=0pt,partopsep=0pt]
    \item \textbf{Principales focos de infección}: las artrosporas infecciosas se presentan en cultivos micelares y muestras de tierra. Las esférulas (cuerpos esféricos de pared gruesa con endosporas en su interior) se presentan en muestras clínicas (esputos y lesiones cutáneas) y tejidos animales infectados.
    \item \textbf{Peligros en el laboratorio}: ampliamente documentado, la infección con esférulas parasitarias (poco infecciosas) se da por la inoculación accidental de pus u otras sustancias, conformando granulomas. La inhalación de artrosporas es la principal causa de enfermedades, encontrandose en muestras de tierra, cultivos micelares o transformación de las esférulas de material clínico.
    \item \textbf{Precauciones recomendadas}: nivel de contención 3 para manipulación de cultivos, tierras o material susceptible de tener artrosporas, así como recomendaciones para evitar aerosoles.
\end{itemize}
\subsubsection{\textit{Cryptococcus neoformans} (o \textit{Filobasidiella neoformans})}
Presente en heces de aves, es un microorganismo frecuente que sólo infecta a organismos inmunodeprimidos. Las manifestaciones de la infección son: infección pulmonar, meningitis, osteomielitis, fungemia, endocarditis, infección cutánea, queratitis micótica, celulitis orbital, infección endoftálmica.
\begin{itemize}[itemsep=0pt,parsep=0pt,topsep=0pt,partopsep=0pt]
    \item \textbf{Principales focos de infección}: las levaduras se encuentran en el suelo en forma no encapsulada, que por su tamaño permiten su inhalación. Se ha encontrado también en materiales ambientales contaminados con excremento de paloma y en muestras clínicas contaminadas.
    \item \textbf{Peligros en el laboratorio}: No se conocen casos de infección respiratoria en laboratorios, sí por vía parenteral. Presentan también peligros la manipulación de cultivos, material infeccioso o mordeduras de animales de experimentación , siguiendo la vía parenteral.
    \item \textbf{Precauciones recomendadas}: nivel de contención 2 para material susceptible de infección. Los trabajos con material ambiental susceptible de tener levaduras de este microorganismo se manipulará en cabina de seguridad biológica.
\end{itemize}
\subsubsection{\textit{Histoplasma capsulatum}  (o \textit{Ajellomyces capsulatus})}
La infección por este microorganismo se presenta con una primoinfección pulmonar con diversas formas clínicas (asintomática, respiratoria aguda benigna, diseminada aguda, diseminada crónica, pulmonar crónica) que avanza a micosis generalizadas de gravedad variable.
\begin{itemize}[itemsep=0pt,parsep=0pt,topsep=0pt,partopsep=0pt]
    \item \textbf{Principales focos de infección}: la forma infecciosa (conidio) se presenta en cultivos celulares esporulantes y tierra, la forma de levadura está en fluidos y tejidos de seres infectados (la inoculación parenteral provoca infección local).
    \item \textbf{Peligros en el laboratorio}:  se derivan de la inhalación de conidios infecciosos, cuyo tamaño (< 5 $\mu$m) facilitan la dispersión aérea y retención pulmonar. Existen datos de infecciones por pinchazos y otras vías parenterales, que generan infecciones locales.
    \item \textbf{Precauciones recomendadas}:  Nivel de contención 3 para cultivos de micelio, muestras de tierras o material sospechoso de estar infectado.
\end{itemize}
\subsubsection{\textit{Sporothix schenckii}}
Genera micosis cutáneas, que comienzan con un nódulo y localizándose en un miembro, paso posterior los ganglios linfáticos que drenan a la región afectada se vuelven duros y se ulceran los nódulos.  Raramente produce artritis, neumonía y otras infecciones viscerales.
\begin{itemize}[itemsep=0pt,parsep=0pt,topsep=0pt,partopsep=0pt]
    \item \textbf{Principales focos de infección}: muestras clínicas procedentes de aspiración de lesiones, pus, exudados y muestras de suelo y vegetales.
    \item \textbf{Peligros en el laboratorio}:  la infección local de piel u ojos se produce por salpicaduras, rasguños o pinchazos con material infectado, o por vía parenteral por mordedura de animal infectado. Las lesiones locales se generan por manipulación de cultivos o autopsias.
    \item \textbf{Precauciones recomendadas}:  nivel de contención 2.
\end{itemize}
\subsubsection{Tiñas: Especies patógenas de \textit{Epidermophyton floccosum}, \textit{Microsporum} spp, \textit{Tricophyton spp}}
Producen micosis cutáneas o tiñas en regiones queratinizadas. La gravedad varía según género y especie del dermatofito. Por regla general,  se produce descamación, perdida del pelo, eritema, induración, costra y supuración, con lesiones circulares.
\begin{itemize}[itemsep=0pt,parsep=0pt,topsep=0pt,partopsep=0pt]
    \item \textbf{Principales focos de infección}:  piel, pelos y uñas de huéspedes.
    \item \textbf{Peligros en el laboratorio}:  La mayor parte se han adquirido en el contacto con animales de laboratorio infectados. Rara vez con manejo de cultivos o material clínico.
    \item \textbf{Precauciones recomendadas}:  nivel de contención 2 para trabajos y experimentos con portadores o presuntos portadores.
\end{itemize}
\subsubsection{Otros mohos patógenos}
Los siguientes mohos generan una micosis crónica llamada cormoblastomicosis, adquirida por inoculación accidental de esporas, empezanco con el desarrollo de una pápula y la extensión en lesiones similares a verrugas o tumores. Las especies patógenas son: \textit{Cladosporium trichoides}, \textit{Cladosporium bantianum}, \textit{Penicillum marnffei}, \textit{Exophiala dermatitis}, \textit{Fonsecaea pedrosoi}, \textit{Dactylaria gollopavum/Ochronosis gallopavum}.
\begin{itemize}[itemsep=0pt,parsep=0pt,topsep=0pt,partopsep=0pt]
    \item \textbf{Principales focos de infección}:  muestras ambientales.
    \item \textbf{Peligros en el laboratorio}:  la inhalación de conidios de cultivos en fase de esporulación y la inoculación accidental son los únicos riesgos descritos para el personal de laboratorio.
    \item \textbf{Precauciones recomendadas}:  nivel de contención 2 en la manipulación de agentes o fases de reproducción.
\end{itemize}
\subsection{Riesgo específico por exposición a parásitos}
El riesgo biológico frente a los parásitos es el de menor difusión, aunque su manipulación es un riesgo si no se realiza en condiciones de seguridad biológica. Estos organismos generan con el hospedador una relación en la que este primero recibe un perjuicio, llamándose parasitismo. Los parásitos pueden ser protozoos unicelulares o seres pluricelulares (nemátodos helmínticos, cestodos (tenias) y termatodos (fasciolas)).
\subsubsection{Fuentes de parasitosis}
\begin{itemize}[itemsep=0pt,parsep=0pt,topsep=0pt,partopsep=0pt]
    \item Por contacto directo a través de otra persona (\textit{Trichomonas vaginalis}).
    \item Por vía oral-fecal (i.e. oxiuros como \textit{Enterobius vermiculatis}).
    \item Vía congénita/maternofilial (\textit{Toxoplasma} spp).
    \item Por fómites (\textit{Enterobius} spp).
    \item Por suelo contaminado con heces infectadas (\textit{Ancylostoma} spp).
    \item Por agua o alimentos contaminados (\textit{Entamoeba histolytica}, \textit{Trichinella spiralis}).
    \item Por animales parasitados (\textit{Echinococcus granulosus}).
    \item Vectores artrópodos (\textit{Plasmodium} spp, vehiculado por mosquitos \textit{Anopheles}).
\end{itemize}
\subsubsection{Vías de entrada}
Los parásitos pueden penetrar por vía cutánea, por contacto con mucosas parasitadas o vía oral. La vía respiratoria es excepcional (la utilizan \textit{Toxoplasma} o \textit{Pneumocystis carinii}). Otra vía de entrada es mediante transfusiones de sangre o vectores (i.e. \textit{Plasmodium}).

Los parasitos protozoarios (\textit{Toxoplasma}, \textit{Plasmodium}, \textit{Trypanosoma}, \textit{Entamoeba}, \textit{Coccidia}, \textit{Giardia}, \textit{Leishmania}, \textit{Sarcicocystis}, \textit{Cryptosporidium}) son escasos para el caso humano, necesitando vivir en un medio líquido. Si no está en un ambiente favorable, forman quistes que son fácilmente diseminables. En estos casos, la infestación se da por pinchazos accidentales o ingestión de quistes, esporas u oocistes en heces.

Aquellos parásitos transmisibles por artrópodos (i.e \textit{Trypanosoma cruzi}), que se encuentran viviendo en su saliva y entran por mordedura o pinchazo del artrópodo. Son transmisibles por vía parenteral o transfusión.

Un gran riesgo son las infecciones por \textit{Cryptosporidium} en personal que maneja animales infectados, en especial con terneras jóvenes infestadas con oocistes.
\begin{itemize}[itemsep=0pt,parsep=0pt,topsep=0pt,partopsep=0pt]
    \item \textbf{Peligros en el laboratorio}: los diferentes estadios infecciosos pueden estar presentes en sangre, heces, lesiones y artrópodos. Una fuente  directa puede ser el contacto con material procedente de lesiones (i.e \textit{Leishmania} en roedores infectados). La entrada del parásito está condicionada al tipo del parásito al que se está expuesto: ingestión, vía parenteral, penetración por heridas, picaduras de artrópodos, exposición a mucosas de ojos, nariz o boca, aerosoles con trofozoitos o formas móviles, o con homogeneizados de tejidos o sangre. Los individuos inmunodeprimidos o embarazadas deben evitar trabajar con organismos vivos.
    \item \textbf{Precauciones recomendadas}: nivel de contención 2, junto con el uso de cabinas de seguridad biológica.
\end{itemize}
\subsection{Riesgos específicos debidos al empleo de animales}
La legislación sobre animales como modelos de experimentación obligan a utilizar el mínimo de animales en el mínimo de usos posibles. Esto se regula en el RD 223/1988 y el Convenio Europeo de 1986.

La investigación puede requerir de animales deliberadamente infectados, con el riesgo de contaminación. Así mismo, los animales de experimentación pueden ser reservorios de enfermedades infecciosas, producir alergias o ser fuente de accidentes por arañazos, mordeduras, picaduras, etc. Su manipulación se regula mediante el RD 664/1997 y la OM 25/marzo/1998 sobre protección de la clase trabajadora.
\subsubsection{Organización y distribución del animalario}
Los animales de experimentación tendrán alojamiento confortable, higiénico y con unas dimensiones suficientes que garanticen cierta libertad o movimiento, además de estar provistos de agua, alimentos y cuidados. El personal encargado debe certificar la correcta salud y condiciones de los animales. Ya sea en el experimento o al final de la vida útil, debe hacerse mediante métodos que supongan el mínimo sufrimiento físico y mental.

El área destinada a la experimentación debe incluir las siguientes zonas:
\begin{itemize}[itemsep=0pt,parsep=0pt,topsep=0pt,partopsep=0pt]
    \item Sala donde se alojan los animales de forma permanente (animalario, estabulario) que debe estar diseñadas en función del animal, riesgo que representa y medidas de protección correspondiente.
    \item  Sala de cuarentena, para la prevención de zoonosis, donde se recepcionan los animales.
    \item Sala de manipulación o laboratorio, donde se llevan a cabo los tratamientos, intervenciones o autopsias. Debe estar equipada para experimentos quirúrgicos asépticos. Así, se recomienda disponer una sala para postoperatorios.
    \item Sala de limpieza para lavado de cajas, camas, jaulas y material.
    \item Almacén y vestuario para personal, que debe ser adyacente.
\end{itemize}
\subsubsection{Riesgos debidos a la manipulación de animales}
Los riesgos se clasifican en:
\begin{itemize}[itemsep=0pt,parsep=0pt,topsep=0pt,partopsep=0pt]
    \item \textit{Riesgos inherentes a los animales}: por ser portadores de enfermedades infecciosas.
    \item \textit{Riesgos resultatas de la investigación realizada}.
    \item \textit{Riesgos generados por animales transgénicos}: estos animales puede ser susceptibles a enfermedades o estado inmunes modificados. Se deben usar niveles de seguridad biológica elevados.
\end{itemize}

Los riesgos al trabajar con estos animales deberían estar limitados por medidas de aprovisionamiento y supervisión veterinaria. Se debe evaluar siempre el riesgo conociendo la especie, las infecciones susceptibles, la naturaleza de agentes infecciosos... Cuanto más alejado filogenéticamente sea la especie frente al ser humano, menor el riesgo de transimisión de patógenos.
\subsubsection{Recomendaciones generales destinadas a la protección laboral}
Todo personal que trabaje con animales debe estar informado de los riesgos inherentes al trabajo y recibir formación en técnicas, instrumentación, metodología y EPIs para evitar contraer la enfermedad e impedir la dispersión de agentes biológicos. Desde el punto de vista estructural, los servicios relacionados con las instalaciones (almacenes, vestuarios y lavabos) deben estar cerca de la unidad animal, aunque fuera, y siempre teniendo en cuenta los niveles de seguridad.

En el trabajo con animales de adoptan criterios generales aplicables a laboratorios y centros de trabajo donde se manipulan agentes biológicos, teniendo en cuenta el tipo de microorganismo con el que se trabaja o posibles cepas portadas.
\subsubsection{Instrucciones para el trabajo según el nivel de seguridad}
\paragraph{Nivel de seguridad 1 (medidas básicas)}
\begin{itemize}[itemsep=0pt,parsep=0pt,topsep=0pt,partopsep=0pt]
    \item \textbf{Infraestructura}: los locales estarán cerrados y protegidos procurando que las salidas al exterior sean las menos posibles, con dispositivos de cerradura automática y permanentemente cerradas. Los techos, paredes y suelos deben ser materiales resistentes, no porosos y fáciles de lavar y desinfectar. El suelo será uniforme, impermeable y antideslizante, capaz de soportar sin peligro el peso y desplazamiento del mobiliario. Los sifones deben ser descontaminados regularmente. Las aberturas deben estar provistas de dispositivos que impidan la entrada de otros seres vivos y, si hay una entrada, se informará al responsable; además de evitar la fuga o salida al exterior (barreras sucesivas, neutralización), siendo sacrificados una vez recapturados (nivel de residuo sanitario no específico, grupo II). En cada local destinado a la instalación de animales debe haber un lavabo para manos y para el lavado de jaulas. El suelo de las jaulas estará siempre limpio y se renovará periódicamente. Las jaulas limpias se guardarán en locales separados. Debe disponerse de locales separados para alimentos y camas, estando en lugares secos y libre de plagas. Según el origen de los animales, se habilitará una zona de cuarentena, variable según la especie, y que será más larga cuanto mayor sea el riesgo de zoonosis. Las muertes o enfermedades inesperadas se notificarán de forma inmediata y no se tocará al animal salvo instrucciones del responsable. El sistema de ventilación será apropiado a las exigencias termohigrométricas de la especie, garantizando 15 renovaciones de aire por hora (dependiendo de las circunstancias, puede ser mayor o menor) para la reducción de gases y olores, circulando del nivel menos contaminado al más contaminado. 
    \item \textbf{Conducta del personal}: el personal entrará en las salas de manipulación con los EPIs adecuados, y unicamente entrarán las personas que realizarán el experimento o autorizados. Se lavará cuidadosamente las manos tras manipular animales (vivos o muertos) y al abandonar el local. Las heridas producidas por animales se tratarán de forma inmediata: se estimulará la hemorragia, se lavará profusamente y aplicará un apósito protector; reiniciándose el tratamiento lo antes posible. Se debe minimizar la generación de aerosoles, así como se prohíbe todo alimento no destinado al consumo animal. Se debe contar la excreción de agentes por saliva, heces, orina y en el limpiado de camas y jaulas. Todo el personal debe estar inmunizado contra el tétanos y otras enfermedades para las que se disponga vacuna. Los animales pueden ser portadores asintomáticos de microbios peligrosos para el ser humano. Se tomarán precauciones especiales para la administración de medicamentos para sedación o eutanasia, con un control estricto, y con al menos una persona debe estar informada de medidas ante autoexposición a inyecciones y anestésicos volátiles.
    \item \textbf{Alojamiento de los animales}: las jaulas, cajas, estantería e instalaciones se construirán de materiales que no representen peligro para los animales y que sea fácilmente desinfectable. Se garantizará un control periódico de los animales, y ante la llegada, se examinará por una persona competente que defina medidas de cuarentena, con autorización del responsable. Los cadáveres y desechos se eliminarán rápidamente según legislación (i.e. en caso de radiomarcaje, se seguirá la reglamentación de residuos radiactivos). En la espera, se guardará en frío en embalajes estancos.
\end{itemize}
\paragraph{Nivel de seguridad 2 }
\begin{itemize}[itemsep=0pt,parsep=0pt,topsep=0pt,partopsep=0pt]
    \item \textbf{Infraestructura}: la unidad estará situado en una zona específicamente reservada, lejos de zonas de paso. El acceso estará concebido de roma que permite al personal cambiarse de ropa para acceder a la unidad y tendrá un autoclave cerca. Las jaulas se desinfectarán mediante autoclave o solución descontaminante eficaz. Las superficies de trabajo se descontaminarán al finalizar el experimento. La señalización internacional de peligro biológico se colocará en las puertas y se podrán cerrar las puertas herméticamente.
    \item \textbf{Conducta del personal}: será obligatorio llevar guantes resistentes a mordeduras o impermeables ante la exposición al material infeccioso sea inevitable. El personal se cambiará de ropa de trabajo y zapatos al entrar y salir de la unidad. Necesario el uso de fundas protectoras y mascarillas. El acceso se limitará al personal informado y que sea imprescindible para la investigación. El médico de la empresa delimitará que personas no están autorizadas.
    \item \textbf{Alojamiento de los animales}: se tomarán precauciones para evitar la agresión por parte de animales durante diferentes fases. Se utilizarán jaulas de contención y se anestesiará al animal antes de cogerlo. Toda manipulación susceptible de generar aerosoles se efectuará en cabinas de seguridad biológica (clase I o II) y protección facial, con énfasis en autopsias y recogida de tejidos infectados o inoculación intranasal. Los residuos se descontaminarán mediante autoclave o cabina de fumigación y se eliminarán  por vías convencionales. Los cadáveres se evacuarán en bolsas de plástico dobles, soldadas y estancas y se eliminarán como residuo sanitario grupo III.
\end{itemize}
\paragraph{Nivel de seguridad 3}
\begin{itemize}[itemsep=0pt,parsep=0pt,topsep=0pt,partopsep=0pt]
    \item \textbf{Infraestructura}: los dispositivos de lavado de manos serán de accionamiento no manual, mediante codo, pie o forma automática, próximos a las salidas. El autoclave se situará en el interior de la unidad animal. Cuando se trate de animales grandes, se dispondrá de un dispositivo de fumigación o baño con desinfectante, con posibilidad de acceso por interior o exterior de la unidad. Se señalizarán los accesos y estos serán de doble puerta. Las ventanas no son practicables y habrá una zona de vestuario con ducha próximas a la salida. El sistema de ventilación debe evacuar el aire al exterior del edificio en una zona donde no exista posibilidad de reciclado o pasado a través de filtros HEPA. El aire circulará desde el exterior al interior, estando el vestíbulo en depresión intermedia. Se utilizará un circuito de vacío con filtros HEPA y trampas de agua con desinfectante apropiado.
    \item \textbf{Conducta del personal}: el vestuario debe ser completo, con protectores de zapatos o botas y máscara filtrante y se debe descontaminar antes de llevar a lavar. Se hará consignas estrictas de descontaminación de las zonas. Será obligatorio el uso de guantes responsables para la manipulación y se descontaminarán en autoclave. 
    \item \textbf{Alojamiento de los animales}: en cada local dedicado a una especie diferente de microorganismo, las jaulas ocupadas por animales infectados se situarán en recintos de seguridad adaptados. El aire filtrado debe expulsarse dentro del circuito de extracción principal del laboratorio o directamente al exterior con garantía de que no haya reciclado. Los cadáveres se colocarán en bolsa doble estaca dentro de la unidad, el exterior se descontaminará por fumigación o sumergiéndola en baño descontaminante. Se eliminarán como residuo sanitario grupo III. Las jaulas se descontaminarán en el autoclave antes del cambio de cama y antes de al limpieza, con precauciones especiales para la manipulación de camas (se cubrirán las cubetas con camas con plástico sobresalidendo por los bordes de forma que se puede tomar por los extremos y pueda depositarse dentro de bolsas de plástico fácilmente), y la descontaminación de la jaula puede hacerse en baño descontaminante dentro de la unidad.
\end{itemize}
\paragraph{Nivel de seguridad 4}
\begin{itemize}[itemsep=0pt,parsep=0pt,topsep=0pt,partopsep=0pt]
    \item \textbf{Infraestructura}: la unidad se situará en un edificio separado o en una zona aislada en el interior del edificio. Las aberturas estarán selladas y será obligatorio un vestuario de doble zona con ducha en cada entrada. Las conducciones eléctricas, luminarias, bocas y conductos de aspiración se concebirán para la mínima deposición de polvo, los circuitos de ventilación serán estancos y las puertas deberán cerrase con llave. Una señal indicará cuando los animales infectados estén presentes en el interior de la misma.  El sistema de ventilación será autónomo, con depresión controlada, con aire circulando de la zona menos a la más contaminada, filtrándose por filtros HEPA y se expulsarán lejos de toda boca aspiración de aire. Un sistema de alarma advertirá de fallos. Deben tomarse precauciones durante el cambio de filtros, con descontaminación previa y empleando ropa de protección. La descontaminación por calor o solución descontaminante de eficacia comprobada para los efluente líquidos (lavabo, aseo, ducha, sifón de baño, condensación del autoclave) y con puerta doble.
    \item \textbf{Conducta del personal}: el acceso a los locales se restringirá a personal autorizado por el responsable, con autorización médica previa y con información de como se debe proteger. El personal entrará y saldrá por el compartimiento estanco del vestuario. Se cambiará la ropa por ropa de trabajo (guardapolvo, pantalón, camisa, calzado y guantes) al entrar al animalario. A la salida, se quitará y se dejará en zona de espera. Será obligatorio ducharse antes de ponerse la ropa de calle. Será obligatorio una vigilancia médica especial para el personal con acceso, teniendo una seroteca del personal. Las normas serán detalladas ante situaciones de emergencia y se informará sobre el cumplimiento de las mismas.
    \item \textbf{Alojamiento de los animales}: los animales infectados se mantendrán en las cabinas de confinamiento de clase III dentro de salas en depresión o dentro de dispositivos de confinamiento parcial, ventilados por una corriente de aire ascendente filtrado por HEPA. Los cadáveres, partes y otros residuos, camas o cualquier material contaminado o procedente de animales inoculados con agentes biológicos de grupo 4 se eliminarán como residuo sanitario de grupo III.
\end{itemize}
\section{Riesgos radiológicos}







